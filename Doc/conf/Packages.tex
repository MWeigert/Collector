%----------------------------------------------------------------------------------------

% Formatvorlage fuer Arbeiten von Mathias Weigert (Grundlage der Datei und allg. Struktur
% von Micha Schoeneberger)

%--------------------------------------VERSION-------------------------------------------

% Version V0.75 - Erstellt von Mathias Weigert am 16.09.2013
%				          Etwas aufgeraeumt, muss mal bei Zeit detailiert Ueberarbeitet werden.
%         V0.80 - Erstellt von Mathias Weigert am 24.08.2015
%                 Kleinere Anpassungen und nochmalieges "aufräumen" der einzelnen
%                 Packages.

%---------------------------------- INFORMATIONEN ---------------------------------------

% 	Definition von globalen Parametern, die im gesamten Dokument verwendet
% 	werden koennen (z.B auf dem Deckblatt etc.).
%----------------------------------------------------------------------------------------

\usepackage{cite}					% Verbesserte Zitierungsmoeglichkeiten
\usepackage{tocbasic}

%----------------------- FORMATIERUNG KOPF- UND FUSSZEILE -------------------------------

\usepackage
[automark,							% Automatische Kopfzeile
%headtopline,						% Linie ueber dem Seitenkopf
%plainheadtopline,					% Plain, Linie ueber dem Seitenkopf
headsepline,						% Linie zwischen Kopf und Textkoerper
ilines,								% Trennlinie linksbuendig ausrichten
%plainheadsepline,					% Plain, Linie zwischen Kopf und Textkoerper
footsepline,						% Linie zwischen Textkoerper und Fuss
plainfootsepline,   				% Plain, Linie zwischen Textkoerper und Fuss
%footbotline,						% Linie unter dem Fuss
%plainfootbotline   				% Plain, Linie unter dem Fuss
]{scrpage2}

%----------------------------------------------------------------------------------------


%-------------------------- ANPASSUNG LANDESSPRACHE--------------------------------------

\usepackage{ngerman}

%----------------------------------------------------------------------------------------


%--------------------------------- UMLAUTE-----------------------------------------------

\usepackage[T1]{fontenc}			% Dadurch ist auch sichergestellt, dass in einem PDF
									% Umlaute gefunden werden.
%\usepackage[utf8x]{luainputenc}
\usepackage[utf8]{inputenc}			% Unterstützung erweiterter Zeichensätze mit
									%  unterschiedlichen Kodierungen
									% (z. B. ae, oe,  usw.)
\usepackage{textcomp} 				% Text Companion Schriften (grosse Symbolbibliothek)

%----------------------------------------------------------------------------------------

\usepackage{pifont}					% PostScript standard Symbol and Dingbats fonts

%--------------------------------EIGENE REFERENZIERUNG-----------------------------------

\newcommand{\refTC}[1]{ \ref{#1} \nameref{#1}  }
% gibt aus z.B.: 1.1.3 Erwartetes Resultat (<Kapitelnummer> <Kapitelname>)

%----------------------------------------------------------------------------------------


%----------------------------------- GRAFIKEN--------------------------------------------

\usepackage[dvips,final]{graphicx}		% Einbinden von EPS-Grafiken [draft oder final]
\graphicspath{{pic/}} 					% Hier liegen die Bilder des Dokuments

%----------------------------------------------------------------------------------------

\usepackage{amsmath,amsfonts}			% Befehle aus AMSTeX fuer mathematische Symbole
										% z.B. \boldsymbol \mathbb
\usepackage{multirow}					% Packages fuer Tabellen

%-------------------------------- OWN COLORS---------------------------------------------

\usepackage{conf/rgbcolor}				% Eigene Farbdefinitionen nach RGB Chart
\usepackage[table]{xcolor}				% Um Tabellen einzuf�rben

%----------------------------------------------------------------------------------------

\usepackage{makeidx}					% Fuer Index-Ausgabe; \printindex

%------------------------------ QUELLCODE AUSGABE----------------------------------------

%************************************************************************************
%*                                                    Definitionen zur Quellcode Darstellung                                                   *
%************************************************************************************

\usepackage{listings}

\lstset {language=C++,
	keywordstyle={\color{Blue}\bfseries},
	commentstyle={\color{ForestGreen}\slshape},
	stringstyle={\color{Purple}},
	backgroundcolor={\color{PowderBlue}},
	showstringspaces=fales,
	stepnumber=2,
	numbers=left,
	numberstyle=\tiny}

\lstset {language=SQL,
	keywordstyle={\color{IndianRed}\bfseries},
	commentstyle={\color{ForestGreen}\slshape},
	identifierstyle=\ttfamily\color{CadetBlue}\bfseries,
	stringstyle={\color{Purple}},
	backgroundcolor={\color{Tan}},
	showstringspaces=fales,
	stepnumber=2,
	numbers=left,
	numberstyle=\tiny}						% Eigene Ausgabeformatierungen fuer Quellcode

%----------------------------------------------------------------------------------------


%-------------------------- ZEILENABSTAENDE UND SEITENRAENDER------------------------------

\usepackage{setspace}					% hat zur Benützung 3 Optionen: [singlespacing,
										% onehalfspacing, doublespacing]
\usepackage{geometry}
\usepackage{lineno}						% ermoeglicht das Numerieren der Zeilen

\usepackage{titlesec}
\titleformat{\chapter}
{\normalfont\huge\bfseries}{\thechapter.}{11pt}{\Huge}
\titlespacing*{\chapter}{0pt}{-40pt}{15pt}


% Symbolverzeichnis --------------------------------------------------------
% Symbolverzeichnisse bequem erstellen, beruht auf MakeIndex.
% makeindex.exe %Name%.nlo -s nomencl.ist -o %Name%.nls
% erzeugt dann das Verzeichnis. Dieser Befehl kann z.B. im TeXnicCenter
% als Postprozessor eingetragen werden, damit er nicht staendig manuell
% ausgefuehrt werden muss.
% Die Definitionen sind ausgegliedert in die Datei Abkuerzungen.tex.
% --------------------------------------------------------------------------
\usepackage[intoc]{nomencl}
  \let\abbrev\nomenclature
  \renewcommand{\nomname}{Abkürzungsverzeichnis}
  \setlength{\nomlabelwidth}{.25\hsize}
  \renewcommand{\nomlabel}[1]{#1 \dotfill}
  \setlength{\nomitemsep}{-\parsep}


%inserted november 2012 / Micha
\usepackage[normalem]{ulem}
\newcommand{\markup}[1]{\uline{#1}}


\usepackage{floatflt}							% Zum Umfliessen von Bildern

\usepackage[hyphens]{url}			% bricht lange URL's um, ohne dass der Link defekt ist
%\usepackage{url}

%-------------------------------PDF OPTIONEN----------------------------------------------

\usepackage[
bookmarks,
bookmarksopen=true,
pdftitle={\titel},
pdfauthor={\autorFirst},
pdfcreator={\autorFirst},
pdfsubject={\titel},
pdfkeywords={\titel},
colorlinks=true,
linkcolor=red, 							% einfache interne Verknuepfungen
anchorcolor=black,						% Ankertext
citecolor=blue, 						% Verweise auf Literaturverzeichniseintraege im Text
filecolor=magenta, 						% Verknuepfungen, die lokale Dateien oeffnen
menucolor=red, 							% Acrobat-Menuepunkte
urlcolor=cyan,
%linkcolor=black, 						% einffdrtext
%citecolor=black, 						% Verweise auf Literaturverzeichniseintraege im Text
%filecolor=black, 						% Verknuepfungen, die lokale Dateien öffnen
%menucolor=black,						% Acrobat-Menuepunkte
%urlcolor=black,
backref,
%pagebackref,
plainpages=false,						% zur korrekten Erstellung der Bookmarks
pdfpagelabels,							% zur korrekten Erstellung der Bookmarks
hypertexnames=false,					% zur korrekten Erstellung der Bookmarks
linktocpage 							% Seitenzahlen anstatt Text im Inhaltsverzeichnis verlinken
]{hyperref}

\usepackage[final]{pdfpages}			%Ermoeglicht andere PDF ins Dokument einzufuegen.

%-----------------------------------------------------------------------------------------

\usepackage{chngcntr}					% Zum fortlaufenden Durchnummerieren der Fussnoten

% fuer lange Tabellen
\usepackage{longtable}
\usepackage{array}
\usepackage{ragged2e}
\usepackage{lscape}

%Spaltendefinition rechtsbaendig mit definierter Breite
\newcolumntype{w}[1]{>{\raggedleft\hspace{0pt}}p{#1}}

% Formatierung von Listen aendern
\usepackage{paralist}
% \setdefaultleftmargin{2.5em}{2.2em}{1.87em}{1.7em}{1em}{1em}

\usepackage{lastpage}				% Dient zum ermitteln der Seiten eines Dokuments

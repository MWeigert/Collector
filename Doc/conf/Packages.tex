%----------------------------------------------------------------------------------------

% Formatvorlage f�r Arbeiten von Mathias Weigert (Grundlage der Datei und allg. Struktur
% von Micha Sch�neberger)

%--------------------------------------VERSION-------------------------------------------

% Version V0.75 - Erstellt von Mathias Weigert am 16.09.2013
%				  Etwas aufgeräumt, muss mal bei Zeit detailiert �berarbeitet werden

%---------------------------------- INFORMATIONEN ---------------------------------------

% 	Definition von globalen Parametern, die im gesamten Dokument verwendet
% 	werden k�nnen (z.B auf dem Deckblatt etc.).
%----------------------------------------------------------------------------------------

\usepackage{cite}					% Verbesserte Zitierungsm�glichkeiten
\usepackage{tocbasic}

%----------------------- FORMATIERUNG KOPF- UND FUSSZEILE -------------------------------

\usepackage
[automark,							% Automatische Kopfzeile
%headtopline,						% Linie über dem Seitenkopf
%plainheadtopline,					% Plain, Linie über dem Seitenkopf
headsepline,						% Linie zwischen Kopf und Textkörper
ilines,								% Trennlinie linksbündig ausrichten
%plainheadsepline,					% Plain, Linie zwischen Kopf und Textkörper
footsepline,						% Linie zwischen Textkörper und Fuss
plainfootsepline,   				% Plain, Linie zwischen Textkörper und Fuss
%footbotline,						% Linie unter dem Fuss
%plainfootbotline   				% Plain, Linie unter dem Fuss
]{scrpage2}

%----------------------------------------------------------------------------------------


%-------------------------- ANPASSUNG LANDESSPRACHE--------------------------------------

\usepackage{ngerman}

%----------------------------------------------------------------------------------------


%--------------------------------- UMLAUTE-----------------------------------------------

%\usepackage[utf8]{inputenc}			% Unterst�tzung erweiterter Zeichens�tze mit
									%  unterschiedlichen Kodierungen
									% (z. B. ä, ö, ü usw.)
\usepackage[T1]{fontenc}			% Dadurch ist auch sichergestellt, dass in einem PDF
									% Umlaute gefunden werden.
\usepackage{textcomp} 				% Text Companion Schriften (grosse Symbolbibliothek)
									
%----------------------------------------------------------------------------------------

\usepackage{pifont}					% PostScript standard Symbol and Dingbats fonts

%--------------------------------EIGENE REFERENZIERUNG-----------------------------------

\newcommand{\refTC}[1]{ \ref{#1} \nameref{#1}  }		
% gibt aus z.B.: 1.1.3 Erwartetes Resultat (<Kapitelnummer> <Kapitelname>)

%----------------------------------------------------------------------------------------


%----------------------------------- GRAFIKEN--------------------------------------------

\usepackage[dvips,final]{graphicx}		% Einbinden von EPS-Grafiken [draft oder final]
\graphicspath{{pic/}} 					% Hier liegen die Bilder des Dokuments

%----------------------------------------------------------------------------------------

\usepackage{amsmath,amsfonts}			% Befehle aus AMSTeX für mathematische Symbole
										% z.B. \boldsymbol \mathbb
\usepackage{multirow}					% Packages für Tabellen
	
%-------------------------------- OWN COLORS---------------------------------------------

\usepackage{conf/rgbcolor}				% Eigene Farbdefinitionen nach RGB Chart 
\usepackage[table]{xcolor}				% Um Tabellen einzuf�rben

%----------------------------------------------------------------------------------------

\usepackage{makeidx}					% Für Index-Ausgabe; \printindex

%------------------------------ QUELLCODE AUSGABE----------------------------------------

\input{conf/code}						% Eigene Ausgabeformatierungen für Quellcode

%----------------------------------------------------------------------------------------


%-------------------------- ZEILENABSTÄNDE UND SEITENRÄNDER------------------------------

\usepackage{setspace}					% hat zur Benützung 3 Optionen: [singlespacing,
										% onehalfspacing, doublespacing]
\usepackage{geometry}	
\usepackage{lineno}						% ermöglicht das Numerieren der Zeilen


% Symbolverzeichnis --------------------------------------------------------
% Symbolverzeichnisse bequem erstellen, beruht auf MakeIndex.
% makeindex.exe %Name%.nlo -s nomencl.ist -o %Name%.nls
% erzeugt dann das Verzeichnis. Dieser Befehl kann z.B. im TeXnicCenter
% als Postprozessor eingetragen werden, damit er nicht ständig manuell
% ausgeführt werden muss.
% Die Definitionen sind ausgegliedert in die Datei Abkuerzungen.tex.
% --------------------------------------------------------------------------
\usepackage[intoc]{nomencl}
  \let\abbrev\nomenclature
  \renewcommand{\nomname}{Abkürzungsverzeichnis}
  \setlength{\nomlabelwidth}{.25\hsize}
  \renewcommand{\nomlabel}[1]{#1 \dotfill}
  \setlength{\nomitemsep}{-\parsep}
  

%inserted november 2012 / Micha
\usepackage[normalem]{ulem}
\newcommand{\markup}[1]{\uline{#1}}


\usepackage{floatflt}							% Zum Umfließen von Bildern


% Zum Einbinden von Programmcode --------------------------------------------
%\usepackage{listings}
%\usepackage{xcolor} 
%\definecolor{hellgelb}{rgb}{1,1,0.9}
%\definecolor{colKeys}{rgb}{0,0,1}
%\definecolor{colIdentifier}{rgb}{0,0,0}
%\definecolor{colComments}{rgb}{1,0,0}
%\definecolor{colString}{rgb}{0,0.5,0}
%\lstset{%
%    float=hbp,%
%    basicstyle=\texttt\small, %
%    identifierstyle=\color{colIdentifier}, %
%    keywordstyle=\color{colKeys}, %
%    stringstyle=\color{colString}, %
%    commentstyle=\color{colComments}, %
%    columns=flexible, %
%    tabsize=2, %
%    frame=single, %
%    extendedchars=true, %
%    showspaces=false, %
%    showstringspaces=false, %
%    numbers=left, %
%    numberstyle=\tiny, %
%    breaklines=true, %
%    backgroundcolor=\color{hellgelb}, %
%    breakautoindent=true, %
%%    captionpos=b%
%}

\usepackage[hyphens]{url}			% bricht lange URL's um, ohne dass der Link defekt ist
\usepackage{url}

%-------------------------------PDF OPTIONEN----------------------------------------------
								
\usepackage[
bookmarks,
bookmarksopen=true,
pdftitle={\titel},
pdfauthor={\autorFirst},
pdfcreator={\autorFirst},
pdfsubject={\titel},
pdfkeywords={\titel},
colorlinks=true,
linkcolor=red, 							% einfache interne Verknüpfungen
anchorcolor=black,						% Ankertext
citecolor=blue, 			% Verweise auf Literaturverzeichniseinträge im Text
filecolor=magenta, 						% Verknüpfungen, die lokale Dateien öffnen
menucolor=red, 							% Acrobat-Menüpunkte
urlcolor=cyan,
%linkcolor=black, 						% einffdrtext
%citecolor=black, 			% Verweise auf Literaturverzeichniseinträge im Text
%filecolor=black, 						% Verknüpfungen, die lokale Dateien öffnen
%menucolor=black,						% Acrobat-Menüpunkte
%urlcolor=black,
backref,
%pagebackref,
plainpages=false,						% zur korrekten Erstellung der Bookmarks
pdfpagelabels,							% zur korrekten Erstellung der Bookmarks
hypertexnames=false,					% zur korrekten Erstellung der Bookmarks
linktocpage 				% Seitenzahlen anstatt Text im Inhaltsverzeichnis verlinken
]{hyperref}

\usepackage[final]{pdfpages}			%Ermöglicht andere PDF ins Dokument einzufügen.

%-----------------------------------------------------------------------------------------

\usepackage{chngcntr}					% Zum fortlaufenden Durchnummerieren der Fußnoten


% Aliase für Zitate
% \defcitealias{WPProzess}{Wikipedia:~Prozess}

%\usepackage{minitoc}

% für lange Tabellen
\usepackage{longtable}
\usepackage{array}
\usepackage{ragged2e}
\usepackage{lscape}

%Spaltendefinition rechtsbündig mit definierter Breite
\newcolumntype{w}[1]{>{\raggedleft\hspace{0pt}}p{#1}}

% Formatierung von Listen ändern
\usepackage{paralist}
% \setdefaultleftmargin{2.5em}{2.2em}{1.87em}{1.7em}{1em}{1em}
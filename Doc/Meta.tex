%----------------------------------------------------------------------------------------

% Formatvorlage für Arbeiten von Mathias Weigert (Grundlage der Datei und allg. Struktur
% von Micha Schöneberger)

%--------------------------------------VERSION-------------------------------------------

% Version V0.75 - Erstellt von Mathias Weigert am 16.09.2013
%				  Bisher ungenützten "Balast" entfernt und zweite Matrikelnummer 
%				  eingefügt.
% Version V0.80 - Erstellt von Mathias Weigert am 15.10.2014
%				  Befehle für SRS (Software Requirements Specification) hinzugefügt.

%----------------------------------- INFORMATIONEN ---------------------------------------

% 	Definition von globalen Parametern, die im gesamten Dokument verwendet
% 	werden können (z.B auf dem Deckblatt etc.).
%-----------------------------------------------------------------------------------------

%=====================================STUDIENARBEIT=======================================
\newcommand{\titel}{Collector}				% Titel des Dokumentes
\newcommand{\untertitel}{Eine App zur Verwaltung von Sammlungen}		% Untertitel des Dokumentes
\newcommand{\art}{Semesterarbeit}			% Art der Arbeit (Bacholor, Seminar...)
\newcommand{\fachgebiet}{Informatik}		% Fachgebiet der Arbeit
\newcommand{\autorFirst}{Mathias Weigert}	% 1. Autor
\newcommand{\autorSecond}{2. AUTOR}			% 2. Autor
\newcommand{\studienbereich}{Informatik}	% Studienbereich
\newcommand{\matrikelnrFirst}{S10966778}	% Martikelnummer 1. Autor
\newcommand{\matrikelnrSecond}{0000'0000}	% Martikelnummer 2. Autor
\newcommand{\erstgutachter}{Michael Reiser}	% Erstgutachter der Arbeit
\newcommand{\zweitgutachter}{2. DOZENT}		% Zweitgutachter der Arbeit
\newcommand{\jahr}{2015}					% Jahr der Arbeit
%==========================================SRS===========================================
\newcommand{\project}{PROJECT}				% Projekt
\newcommand{\organisation}{ORGANISATION}	% Organisation
\newcommand{\version}{VERSION}				% Version

%-----------------------------------------------------------------------------------------

% Abkürzungen mit korrektem Leerraum

\newcommand{\vgl}{Vgl.\ }
\newcommand{\ua}{\mbox{u.\,a.\ }}
\newcommand{\zB}{\mbox{z.\,B.\ }}
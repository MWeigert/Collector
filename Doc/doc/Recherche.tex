\chapter{Recherche}
\label{ch:recherche}

Die komplette Recherche findet im Google App Store \href{https://play.google.com/store/apps}{GooglePlay} statt. Für die nachfolgende Ist-Analyse werde ich noch auf die einzelne Apps aus dem Apple App Store verweisen, welche ich selbst einsetze und welche mir im Bezug auf Funktionalität und Design von \emph{Collector} die ein oder andere Idee geliefert haben.

\section{My books}

Diese App dient der Inventarisierung und Organisation der persönlichen Bibliothek. Es ist einfach die komplette Bücher Sammlung zu speichern. Bücher können einfach, durch scannen des Barcode hinzugefügt werden.\\

Features:
\begin{itemize}
	\item Hinzufügen eines Buches durch scannen des Barcode
	\item Hinzufügen eines Buches durch eingeben des ISBN Code
	\item Hinzufügen eines Buches durch eingeben von Titel, Autor, etc...
	\item Ansicht aller Bücher eines Autors
	\item Im- und Export aller Daten in ein File
\end{itemize}

Durch eine Vielzahl von Listen kann der Bücher Katalog organisiert werden.
\begin{itemize}
	\item Bücher welche ich kaufen möchte
	\item Bücher welche ich nochmals lesen möchte
	\item Bücher welche mir gefallen haben
	\item Bücher welche ich verliehen habe
\end{itemize} 

Es gibt eine Vielzahl von Filtermöglichkeiten, wie Genre, Autor, Beurteilung, Leser und viele weiteren.\cite{MyBooks}

\section{Meine Bücher Pro}

Sind Sie ein unersättlicher Book-Reader? Haben Sie Schwierigkeiten, den Überblick über alle Ihre Bücher? Verwalten Sie Ihre Buch-Katalog mit diesem App. Behalten Sie Ihre Büchersammlung zusammen mit persönlichen Bewertungen und Anmerkungen, die Sie mit anderen teilen oder privat halten können. Prüfen Sie, ob Sie bereits ein Buch über das Telefon, während Sie einkaufen sind.\cite{MBP} \\

\begin{itemize}
	\item Verwenden Sie die Bulk-Barcode-Scan-Option, um Ihre Bücher durch schnelles Scannen des Barcodes auf der Titelseite mit der Kamera des Telefons eingeben.
	\item Denken Sie daran, die Bücher, die Sie kaufen wollen, mit der Wunschliste. • Export Ihr Buch info im CSV-Format als ein Tabellenkalkulationsprogramm auf Ihrem Computer zu öffnen.
	\item Nehmen Sie ein Bild von dem Buchumschlag mit Ihrem Telefon zu gehen zusammen mit Ihrem Buch Rating und beachten.
	\item Sichern Sie Ihre Buchkatalog Daten an den Server. Wenn Sie Ihr Telefon verlieren oder ein Upgrade auf ein neues Handy, nur importieren Sie Ihre Sammlung auf das neue Mobiltelefon, so dass Sie nicht haben, um wieder in alles.
	\item Track Statistiken über Ihre Bücher wie Gesamtzahl der Bücher Tracked, Average Rating, am häufigsten gelesen, Anzahl der Bücher von Genre verfolgt.
\end{itemize}

\section{Codex}

Codex ist eine Bücherverwaltungs-Anwendung, die Ihnen hilft, die Bücher Ihrer Bibliothek zu verwalten und zu katalogisieren. Alles, was Sie dazu tun müssen, ist, die ISBN des Buchs zu scannen und die Information zum Buch wird aus dem Web heruntergeladen.\\
Sie können Ihre Bücher ordnen, Ihre verliehenen Bücher verwalten und sogar eine Bücher-Wunschliste anlegen, die Ihnen hilft, die günstigsten Preise für die Bücher, die Sie kaufen wollen, zu ermitteln und sich daran zu erinnern, wo sie sie gefunden haben.\\
Teilen Sie Bücher mit Ihrem Freund / Ihrer Freundin über eine einfache SMS an Leute in Ihrer Kontaktliste. Codex holt sich alle Informationen, die Ihre Freunde über das Buch haben müssen.\\
Alle Buchinformationen können über das Internet heruntergeladen werden, einschließlich Buchtitel-Bild. Falls kein Buchtitel-Bild verfügbar ist, können Sie ein Bild aus Ihrem Telefonspeicher als Titelbild auswählen oder ein Foto des Buchs machen.\\
Codex unterstützt auch mehrere Sprachen, nämlich Englisch, Portugiesisch, Polnisch Französisch und Deutsch.\\

Mit Codex können Sie:
\begin{itemize}
	\item Bücher einfach zu Ihrer Liste hinzufügen, indem Sie den ISBN-Barcode des Buchs scannen. Die Daten zum Buch werden aus dem Internet heruntergeladen.
	\item Ein Buch einfach hinzufügen, indem Sie die ISBN eintippen. Wenn kein Barcode verfügbar ist können Sie die ISBN immer manuell eingeben.
	\item Ein Buch hinzufügen, indem Sie die Buchinformationen eingeben. Falls das Buch keinen Barcode hat oder es nicht möglich ist, Buchinformationen im Web aufzufinden, können sie manuell eingegeben werden.
	\item Ihre verliehenen Bücher verwalten. Jedesmal wenn Sie ein Buch verleihen, hilft Ihnen Codex, sich daran zu erinnern, indem es Ihnen ermöglicht, das Buch als verliehen zu markieren und dazu die Person, der Sie es ausgeliehen haben, aus Ihren Kontakten einzutragen.
	\item Eine Wunschliste mit den Büchern, die Sie kaufen wollen, erstellen. Ein Buch, das Sie nicht besitzen, hinzufügen und das Geschäft, in dem Sie es gefunden haben. Jedes Buch kann mehreren Geschäften zugeordnet werden, so daß Sie Preise vergleichen und den günstigsten Preis suchen können.
	\item Sich an die Lage des Geschäfts erinnern, wo Sie die Bücher in der Wunschliste gefunden haben. Falls Sie die Adresse nicht wissen, kann Codex für Sie die GPS-Lokalisation vornehmen.
	\item Ihre Bücher nach Autor, Verlag oder Kategorie filtern.
	\item Leicht Bücher Ihrer Wunschliste oder von Ihnen verliehene Bücher finden.
	\item Ein Bild aus dem Bildspeicher Ihres Telefons auswählen, um es als Buchtitel-Bild zu verwenden. Wenn Sie kein Bild Ihres Buchtitels haven, fertigen Sie einfach ein Foto davon an.
	\item Wollen Sie dieses Buch mit einem Freund teilen? Empfehlen Sie das Buch einfach. Mit Codex können Sie ein Buch über eine einfache SMS-Mitteilung empfehlen. Wenn Ihr Freund / Ihre Freundin auf seinem Telefon Codex ebenfalls hat, wird Codex alle Buchinformationen zusammenstellen, die Ihr Freund / Ihre Freundin benötigt.
	\item Unterstützung in mehreren Sprachen, nämlich Englisch, Portugiesisch, Französisch und Deutsch erhalten.
	\item Bücher in den Formaten CSV und XML importieren und exportieren, so daß sie zwischen Geräten synchronisiert werden können.
	\item Büchersammlungen durchsuchen, um leicht das richtige Buch zu finden.
	\item Wenn Sie ein Buch verleihen, können Sie eine Rückgabefrist eingeben und Codex wird Sie dann erinnern, wenn die Rückgabefrist erreicht ist. Codex wird eine Verleih-Chronik für jedes Buch unterhalten, so daß Sie wissen, wem Sie es geliehen hatten.
\end{itemize}

Codex ist kostenlos, Sie können es solange Sie wollen ohne jegliche Einschränkungen nutzen. Wenn Sie irgendwelche Anregungen haben, kontaktieren Sie bitte das Entwickler-Team über codex.android@gmail.com.\cite{Codex}

\section{Movie Collection}

Mithilfe der Hauptanwendung 'Movie Collection' ist es möglich die eigene Filmsammlung zu katalogisieren und zu verwalten. Nach dem Hinzufügen neuer Titel werden detaillierte Informationen geladen und in der lokalen Datenbank gespeichert.\cite{MC} \\

Folgende Features sind in der Hauptanwendung bereits enthalten:\\
\begin{itemize}
	\item Detaillierte Informationen zu Filmen wie Inhalt, Darsteller, Poster, Laufzeit, Regisseur ...
	\item Detaillierte Informationen zu Serien mit Staffel und Episoden informationen. (Es muss eine separate Liste für Serien in der Listenverwaltung angelegt werden. Anschließend können zu der neuen Liste Serien hinzugefügt werden)
	\item Verknüpfung von Film-Trailern. Ermöglicht das Schauen der Trailer über YouTube-App
	\item Hinzufügen von Filmen/Serien durch die Suche nach Titel / Schlagwort.
	\item Hinzufügen von Filmen/Serien durch das Scannen des Barcodes des Filmes (Schneller IN-APP Barcode-Scanner. Keine separate Applikation nötig).
	\item Hinzufügen eigener/privater Filme inkl. eigener Cover, Beschreibung etc.
	\item Vollwertige Listenverwaltung (es können so viele Listen erstellt werden wie man benötigt. (z.B. separate Listen für Filme, Serien, Musik DVDs, Wunschliste usw.)
	\item Verschieben/Kopieren von Filmen zwischen den Listen.
	\item Markieren von Filmen/Serien-Staffeln als verliehen.
	\item Listen- und Kachelansicht.
	\item Freie Anpassung der Cover/Postergröße in Listen- und Kachelansicht.
	\item Ansprechendes Look and Feel.
	\item Holo-Light und Holo-Dark Theme.
	\item Unterschiedliche Sortierungs/Filter-Optionen.
	\item Filme mit Zeitstempel als \glqq Gesehen\grqq \ markieren.
	\item Suchen in der eigenen Datenbank.
	\item Backup-Manager zum Sichern und Wiederherstellen der internen Datenbank (Export/Import in verschiedenen Formaten möglich wie Binär/XML/CSV).
	\item Sichern der Backups auf dem externen Speicher/SD-Karte. Hochladen der Backups in die Cloud (Dropbox/Box/Google-Drive) oder Versenden an Freunde per Mail etc.
	\item Anzeige der Poster/Backdrops in Vollbildmodus.
	\item Anzeige der Filmdaten in der IMDB Applikation oder der Webpräsenz.
	\item Keine Registrierung notwendig.
	\item Kompatibel zu Android 3.0 \glqq Honeycomb\grqq.
\end{itemize}

\section{Movielicious}

Hast Du eine grosse Blu-ray oder DVD Sammlung? Movielicious ist eine einfache aber mächtige App welche dir hilft deine Filmsammlung zu organisieren. Mit Movielicious kannst du schnell Filme zu deiner Sammlung hinzufügen, finden und bearbeiten. Neue Filme können durch durchsuchen von verschiedenen Quellen, durch benutzen von Freitextsuche, oder durch scannen des Barcodes auf der DVD oder Blu-ray Hülle, hinzugefügt werden. Du kannst die Filme in deiner Sammlung oder Wunschliste organisieren.\cite{Movielicious}

\begin{itemize}
	\item Honeycomb Tablet Unterstützung
	\item IMDb Integration
	\item Wunschliste
	\item Export Sammlung oder Wunschliste via E-Mail, etc.
	\item Hinzufügen von Filmen via scannen des Barcode
	\item Hinzufügen von Filmen via Onlinesuche in verschiedenen Quellen
	\item Hinzufügen von Filmen via manuelle Eingabe
	\item Regal- oder Listenansicht der Filmbibliothek
	\item Suche in der Sammlung nach Film, Titel, Jahr oder Schauspieler
	\item Filter deine Sammlung nach Genre, Regisseur, Format, verliehen oder gesehen.
	\item Teile deine Lieblingsfilme via Facebook, Twitter oder Google+
	\item Bewerte Filme
	\item Stelle fest, wann du diesen Film das letzte mal gesehen hast
	\item Behalte den Überblick, welche Filme du an wen verliehen hast
	\item Funktioniert mit allen modernen Formaten (DVD, Blu-ray, HD-DVD) wie auch mit älteren Formaten (VHS, 35mm, 16mm), zusätzlich können noch eigene Formate hinzugefügt werden
	\item Verlinkung mit der IMDb Android App
	\item Sichere oder Stelle deine Filmbibliothek wieder her
	\item Import der Filmbibliothek aus der iOS Version von Movielicious
	\item Import aus einer Vielzahl von Quellen, inklusvie Textfile oder CSV
\end{itemize}

\section{Nintendo Collection}

We lovers of a time when games were true to their cause have been poorly served when it comes to a dynamic and modern way of cataloguing our trusty old games.\\

This stops now.\\

This application allows you to keep your games collection by your side at all times.\\

Initially this only caters for the NES to gauge the response from you. If you like the premise, your feedback will be taken on board to improve the application and to include more systems.\\

Also currently the box arts are hit and miss and so a more robust alogirthm to retrieve them is being constructed.\cite{NC}

\section{Video Games Manager Collector}

PS4, Xbox One, Wii U, PS3, 360, Nintendo 3DS, PlayStation Network, PS2, Vita, DS, und mehr Videospiele verwalten mit dieser Datenbank. Diese App bietet zu ihrer Sammlung Katalog, Verwaltung und Bibliothek.\\

Verwendet UPC Barcode-Scanning, Datenbanksuche und manuelle Barcode-Eingabe, sowie eine Vielzahl von unglaublichen Funktionen wie Android-, iOS- und Website-Synchronisierung, Wunschlisten, Filtern, Sortieren, Bibliotheksansichten, Bulk Hinzufügen und vieles mehr. Die App zeigt Informationen, Titelbild, und persönliche Daten an. Ideal für das Sammeln.\cite{VGMC} \\

Funktioniert für Videospiele von diesen Plattformen:\\

\begin{tabular}{lll}
	Playstation 4 & Xbox One & Nintendo Wii U \\
	Playstation 3 & Xbox 360 & Nintendo Wii \\
	Playstation 2 & Xbox Live & Nintendo 3DS \\
	Playstation Vita & Xbox & Nintendo DS \\
	Playstation Network & & \emph{und viele mehr!} \\
\end{tabular}

\newpage

\begin{landscape}
	\section{Überblick} 
	\begin{tabular}{|l|c|c|c|c|c|c|r|c|}
		\rowcolor{black} 
		\color{white}\textbf{Name} & \color{white}\textbf{Bücher} & \color{white}\textbf{Filme} & \color{white}\textbf{Spiele} & \color{white}\textbf{Scanner} & \color{white}\textbf{Export} & \color{white}\textbf{Verleih} & \color{white}\textbf{Preis} & \color{white}\textbf{Rating} \\
		My books & Ja & \color{red}Nein & \color{red}Nein & Ja & Ja & Ja & Gratis & 3.7 \\ \hline
		\rowcolor{DarkSeaGreen} Meine Bücher Pro & Ja & \color{red}Nein & \color{red}Nein & Ja & CSV & \color{black}Nein & CHF 3.65 & 4.0 \\ \hline
		Codex & Ja & \color{red}Nein & \color{red}Nein & Ja & CSV, XML & Ja & Gratis & 4.2 \\ \hline
		\rowcolor{DarkSeaGreen} Movie Collection & \color{red}Nein & Ja & \color{red}Nein & Ja & Binär, CSV, XML & Ja & CHF 2.15 & 4.5 \\ \hline
		Movielicious & \color{red}Nein & Ja & \color{red}Nein & Ja & Ja & Ja & CHF 2.50 & 3.2 \\ \hline
		\rowcolor{DarkSeaGreen} Nintendo Collection & \color{red}Nein & \color{red}Nein & Ja (NES) & \color{red}Nein & \color{red}Nein & \color{red}Nein & CHF 1.47 & 5.0 \\ \hline
		Video Games Manager Collector &  \color{red}Nein & \color{red}Nein & Ja & Ja & Nein & Nein & Gratis & 3.6 \\ \hline
	\end{tabular} 
\end{landscape} 
\chapter{Recherche}

Die komplette Recherche findet im Google App Store \href{https://play.google.com/store/apps}{GooglePlay} statt. Für die nachfolgende Ist-Analyse werde ich noch auf die einzelne Apps aus dem Apple App Store verweisen, welche ich selbst einsetze und welche mir im Bezug auf Funktionalität und Design von \emph{Collector} die ein oder andere Idee geliefert haben.

\section{My books}

Diese App dient der Inventarisierung und Organisation der persönlichen Bibliothek. Es ist einfach die komplette Bücher Sammlung zu speichern. Bücher können einfach, durch scannen des Barcode hinzugefügt werden.\\

Features:
\begin{itemize}
	\item Hinzufügen eines Buches durch scannen des Barcode
	\item Hinzufügen eines Buches durch eingeben des ISBN Code
	\item Hinzufügen eines Buches durch eingeben von Titel, Autor, etc...
	\item Ansicht aller Bücher eines Autors
	\item Im- und Export aller Daten in ein File
\end{itemize}

Durch eine Vielzahl von Listen kann der Bücher Katalog organisiert werden.
\begin{itemize}
	\item Bücher welche ich kaufen möchte
	\item Bücher welche ich nochmals lesen möchte
	\item Bücher welche mir gefallen haben
	\item Bücher welche ich verliehen habe
\end{itemize} 

Es gibt eine Vielzahl von Filtermöglichkeiten, wie Genre, Autor, Beurteilung, Leser und viele weiteren.
\section{Funktionen}

Die App Collector soll den folgenden Beschriebenen Funktionsumfang besitzen. Dieser wurde anhand der Recherche, der Ist-Analyse und dem Bedarf von M. Weigert ermittelt. Probleme bei der Umsetzung der Funktionen und deshalb nötige Anpassungen werden im Kapitel \ref{ch:Umsetzung} ab Seite \pageref{ch:Umsetzung} dokumentiert.

\subsection{Anlegen eines Items}

Das Anlegen eines neuen Items soll auf zweierlei Arten Möglich sein.

\begin{enumerate}
	\item Manuell
	\item Automatisch
\end{enumerate}

Nach der Auswahl {\color{IndianRed}\texttt{Neues Item}} wird standardmässig der Screen für das automatische Anlegen eines Items, mit aktivierter Kamera, geöffnet. Auf diesem Screen gibt es einen Button für das manuelle Anlegen eines Items.

\subsubsection{Manuelles Anlegen eines Items}

Sobald der User {\color{IndianRed}\texttt{manuell}} ausgewählt hat öffnet sich der Screen für Bearbeiten und Manuelles anlegen eines Items. Da es sich hierbei um die neu Anlage eines Items handelt ist der Screen noch komplett ohne angezeigte Daten.\\

Nun kann der User alle Daten, welche ein Item ausmachen manuell eingeben und speichern. Um welche Daten es sich hierbei handelt ist in diesem Kapitel im Absatz \ref{sec:Felder} ab Seite \pageref{sec:Felder} dokumentiert.

\subsubsection{Automatisches Anlegen eines Items}

\subsection{Ausgabe aller Items}

\subsection{Bearbeiten eines Items}

\subsection{Löschen eines Items}

\subsection{Exportieren aller Items}

\subsection{Verwalten verliehener Items}

\subsection{Speichern der Items} 
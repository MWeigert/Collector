\section{Funktionen}

Die App Collector soll den folgenden Beschriebenen Funktionsumfang besitzen. Dieser wurde anhand der Recherche, der Ist-Analyse und dem Bedarf von M. Weigert ermittelt. Probleme bei der Umsetzung der Funktionen und deshalb nötige Anpassungen werden im Kapitel \ref{ch:Umsetzung} ab Seite \pageref{ch:Umsetzung} dokumentiert.

\subsection{Anlegen eines Items}

Das Anlegen eines neuen Items soll auf zweierlei Arten Möglich sein.

\begin{enumerate}
	\item Manuell
	\item Automatisch
\end{enumerate}

Nach der Auswahl {\color{IndianRed}\texttt{Neues Item}} wird standardmässig der Screen für das automatische Anlegen eines Items, mit aktivierter Kamera, geöffnet. Auf diesem Screen gibt es einen Button für das manuelle Anlegen eines Items.

\subsubsection{Manuelles Anlegen eines Items}

Sobald der User {\color{IndianRed}\texttt{manuell}} ausgewählt hat öffnet sich der Screen für Bearbeiten und Manuelles anlegen eines Items. Da es sich hierbei um die neu Anlage eines Items handelt ist der Screen noch komplett ohne angezeigte Daten.\\

Nun kann der User alle Daten, welche ein Item ausmachen manuell eingeben und speichern. Um welche Daten es sich hierbei handelt ist in diesem Kapitel im Absatz \ref{sec:Felder} ab Seite \pageref{sec:Felder} dokumentiert.

\subsubsection{Automatisches Anlegen eines Items}

Das automatische Anlegen eines Items ist in der App als Standard, für das Anlegen eines Items definiert. Sobald der User {\color{IndianRed}\texttt{Neues Item}} ausgewählt hat öffnet sich die Kamera. Der User muss nun nur noch den Barcode scannen (photographieren) und eine automatische Suche, im Internet, nach Daten zu diesem Barcode wird gestartet. Die gefundenen Daten werden dem User angezeigt und er kann diese nun bestätigen oder noch anpassen.

\subsection{Ausgabe aller Items}

Wählt der User {\color{IndianRed}\texttt{Sammlung}} aus wird der Filterscreen angezeigt. Je nach Eingabe einzelner Filterkriterien wird die Sammlung nach Ergebnissen durchsucht und diese anschliessend als Liste dem User angezeigt. Werden keine Filterkriterien ausgewählt, so wird die komplette Sammlung als Liste angezeigt.

\subsubsection{Filterfunktion}

Der Screen für den Filter ist in drei verschiedene Reiter unterteilt.

\begin{enumerate}
	\item Allgemein
	\item Buch \& Spiel
	\item Film
\end{enumerate}

Der Reiter {\color{IndianRed}\texttt{Allgemein}} lässt folgende Dateneingabe zu.

\begin{itemize}
	\item Barcode \\
		Hier kann nach einem Barcode gesucht werden.
	\item Titel \\
		Eingabe eines Strings, welcher in allen Titeln der Sammlung gesucht wird.
	\item Medientyp \\
		Definition nach einem oder mehreren Medientypen gesucht werden soll.
	\item Genre \\
		Auswahl eines oder mehreren Genre welche gesucht werden sollen.
	\item Sprache \\
		Auswahl einer oder mehreren Sprachen welche gesucht werden sollen.
	\item Jahr \\
		Auswahl eines Erscheinungsjahr.
	\item Verliehen \\
		Auswahl ob nach Verliehenen Items gesucht werden soll.
\end{itemize} 

\subsection{Bearbeiten eines Items}

\subsection{Löschen eines Items}

\subsection{Exportieren aller Items}

\subsection{Verwalten verliehener Items}

\subsection{Speichern der Items} 
\section{GUI}

\subsection{Allgemeines}

Damit die App auf möglichst vielen Android Geräten, mit verschiedener Bildschirmgrösse, mit einer gleichbleibend hohen Useability funktioniert. Wird beim Layout der App auf folgende Punkte wert gelegt.\\

Die App soll eine einfache und Flache Hierarchie bieten, in welcher der User sich schnell und vor und zurück bewegen kann.\\
Zusätzlich sollen alle GUI Elemente einen relativen Bezug zueinander und zur Bildschirmgrösse haben. So das diese immer an der selben Position den maximalen zur Verfügung stehenden Platz ausnutzen.\\
Um die App schlank und leicht wartbar zu halten, wird wo es geht das Layout für verschiedene Zwecke verwendet. Schaltflächen die bei solch einer Mehrfachverwendung, im speziellen Fall keine Funktion haben, werden automatisch ausgeblendet.\\
Da jedes Android Gerät einen Hardware Zurück Knopf besitzt, ist die GUI der App darauf angelegt das der User diesen Benutzt und stellt selbst keinen Zurück Knopf zur Verfügung. An stellen wo der User eine Handlung abschliesst wird die App automatisch auf das vorherige Layout zurück springen.

\subsection{Hauptmenu / Startscreen}

Damit der User einfach und schnell Zugriff auf alle wichtigen Funktionen der App erhält befinden sich nur fünf Knöpfe auf dem Start- bzw. Hauptlayout. Vier sind davon am oberen Bildschirmrand angeordnet und einer am unteren Bildschirmrand.\\

Von oben nach unten handelt es sich um Knöpfe mit den folgenden Funktionen.

\begin{itemize}
	\item Neues Item
	\item Sammlung
	\item Leihwesen
	\item Export Database
	\item Info
\end{itemize} 

\subsection{Information}

\subsection{Settings}
\section{Datenbank}

[201001] Datenbank: Grunddesign

Die Anforderungen der App an die Datenbank sind einfach und leicht überschaubar. Eine vollständige Übersicht der Tabellen und Felder ist im Abschnitt \ref{subsec:UebersichtDB} zu finden. Um eine einfach zu erweiternde und leicht zu wartende Datenbank zu schaffen, werden für immer wiederkehrende Daten einzelne Tabellen verwendet.

\subsection{Übersicht der Tabellen und Felder}
\label{sec:Felder}

Die SQLITE Datenbank soll folgende Daten zu den einzelnen Items speichern.

\begin{table} [htbp]
	\begin{center}
		\begin{tabular}{|l|l|l|l|}
			\rowcolor{black} {\color{white}\textbf{Allgemein}} & {\color{white}\textbf{Bücher}} & {\color{white}\textbf{Filme}} & {\color{white}\textbf{Spiele}} \\
			Barcode & Verlag & Studio & Entwickler\\ \hline
			\rowcolor{DarkSeaGreen} Titel & Autor & Speichermedium & System \\ \hline		
			Medientyp & & Regisseur & FSK \\ \hline
			\rowcolor{DarkSeaGreen} Genre & & FSK & \\ \hline
			Sprache & & & \\ \hline
			\rowcolor{DarkSeaGreen} Erscheinungsjahr & & & \\ \hline
			Bewertung & & & \\ \hline
			\rowcolor{DarkSeaGreen} Verleihstatus & & & \\ \hline
		\end{tabular}
	\caption{Datenbankfelder}
	\label{tab:Datenbankfelder}
	\end{center}
\end{table}

Die Verleih Verwaltung soll folgende Felder beinhalten.

\begin{itemize}
	\item ID des Freundes
	\item ID des Items
	\item Start des Leihvorgangs
	\item Rückgabe des Items
\end{itemize} 

Um die Datenbank möglichst schlank und übersichtlich zu halten werden folgende Tabellen und Felder angelegt.

\begin{itemize}
	\item Freunde \\
		ID (Key), Vorname, Nachname   
	\item Autoren \\
		ID (Key), Autor
	\item Regisseure \\
		ID (Key), Regisseur
	\item Genres \\
		ID (Key), Genre
	\item Sprachen \\
		ID (Key), Sprache
	\item Verlage \\
		ID (Key), Verlag
	\item Studios \\
		ID (Key), Studio
	\item Systeme \\
		ID (Key), System
\end{itemize}

Die beiden grösseren Tabellen Items und die der Verleih Verwaltung sollen um Speicherplatz zu sparen nur die ID's der einzelnen Daten enthalten.
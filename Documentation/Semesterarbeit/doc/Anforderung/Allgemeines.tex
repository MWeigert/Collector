\section{Allgemeines}

\subsection{Dokumentation der Anforderungen}

Die Anforderungen werden um Mehrdeutigkeit und offene Definitionen zu vermeiden nach folgenden zwei Modellen der Anforderungsanalyse

\begin{itemize}
	\item Natürliche Sprache
	\item Use Case
\end{itemize}

dokumentiert. \\
Bei komplexeren Anforderungen, welche durch die beiden verwendeten Modelle nicht eindeutig dokumentiert werden können, wird zur Spezifizierung noch ein drittes UML Diagramm verwendet. \\

Das dritte Modell wird aus den Notationen von UML 2.3 so ausgewählt, dass die Anforderung Eindeutig beschrieben werden kann.

\subsubsection{Natürliche Sprache}

Um die Mehrdeutigkeit und den möglichen Interpretationsspielraum der natürlichen Sprache zu minimieren, muss die Dokumentation der Anforderung in natürlicher Sprache folgenden Regeln genügen.

\begin{itemize}
	\item Anforderungen immer anhand der Prozesse erklären.
	\item Jeder Prozess einer Anforderung muss eindeutig beschrieben sein.
	\item Substantive, welche zur Anforderungs- oder Prozess Beschreibung eingesetzt werden, müssen über den Bezug für den Leser eindeutig sein.
	\item Bei Mengen und Häufigkeiten nur folgende Quantoren benutzen.
	\begin{itemize}
		\item immer / nie
		\item jeder / kein
		\item alle / irgendein(er) / nichts
	\end{itemize}
	\item Bedingungen immer vollständig dokumentieren.
	\item Nur den sprachlichen Aktiv verwenden.
\end{itemize}

\subsubsection{Use Case }




\subsection{Systemanforderungen}
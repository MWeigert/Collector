\section{Allgemeines}

\subsection{Dokumentation der Anforderungen}

Die Anforderungen werden um Mehrdeutigkeit und offene Definitionen zu vermeiden nach folgenden Modellen der Anforderungsanalyse

\begin{itemize}
	\item Natürliche Sprache
	\item Aktivitätsdiagramm
	\item Use Case
\end{itemize}

dokumentiert. \\
Bei komplexeren Anforderungen, welche durch die verwendeten Modelle nicht eindeutig dokumentiert werden können, werden zur Spezifizierung noch weitere UML Modelle eingesetzt. \\

Diese weiteren Modelle wird aus den Notationen von UML 2.3 so ausgewählt, dass die Anforderung Eindeutig beschrieben werden kann.

\subsubsection{Natürliche Sprache}

Um die Mehrdeutigkeit und den möglichen Interpretationsspielraum der natürlichen Sprache zu minimieren, muss die Dokumentation der Anforderung in natürlicher Sprache folgenden Regeln genügen.

\begin{itemize}
	\item Anforderungen immer anhand der Prozesse erklären.
	\item Jeder Prozess einer Anforderung muss eindeutig beschrieben sein.
	\item Substantive, welche zur Anforderungs- oder Prozess Beschreibung eingesetzt werden, müssen über den Bezug für den Leser eindeutig sein.
	\item Bei Mengen und Häufigkeiten nur folgende Quantoren benutzen.
	\begin{itemize}
		\item immer / nie
		\item jeder / kein
		\item alle / irgendein(er) / nichts
	\end{itemize}
	\item Bedingungen immer vollständig dokumentieren.
	\item Nur den sprachlichen Aktiv verwenden.
\end{itemize}

\subsection{Systematik der Anforderungen}

Jede Anforderung erhält eine eindeutige sechsstellige Anforderungsnummer. Diese leitet sich aus folgenden drei Bestandteilen ab.

\begin{enumerate}
	\item Art der Anforderung (Funktionelle, Datenbank, GUI, Sonstige)
	\item Priorität (1 bis 4)
	\item Laufende eindeutige Nummer
\end{enumerate} 

\subsubsection{Art der Anforderung}

Zur Gliederung wird eine zweistellige Nummer verwendet.

\begin{table} [htbp]
	\begin{tabular}{r|l}
		\textbf{Nr.} & \textbf{Art der Anforderung} \\ \hline
		10 & Anforderung an die Funktion der App \\
		\rowcolor{DarkSeaGreen} 20 & Anforderung an die Datenbank \\
		30 & Anforderung an das GUI \\
		\rowcolor{DarkSeaGreen} 40 & Sonstige Anforderungen
	\end{tabular}
\end{table}

\subsubsection{Priorität der Anforderung}

Die Priorität der Anforderung wird in Ziffern 1 (höchste) bis 4 (niedrigste) definiert.

\begin{enumerate}
	\item Grundlage (bzw. absolutes Muss)
	\item Muss für Grundlegende Funktionen der App
	\item Steigerung der Usability für den Nutzer
	\item Wäre eine tolle zusätzliche Funktion.
\end{enumerate}

\subsubsection{Beispiel: Nummerierung und Bezeichnung der Anforderung}

Die Anforderung  \glqq\emph{Datenbank Design}\grqq \ hat folgende Anforderungsnummer (ID):\\

	\begin{tabular}{c|c|c}
		\textbf{Art} & \textbf{Prio} & \textbf{lfd. Nummer} \\
		\emph{2-stellig} & \emph{1-stellig} & \emph{3-stellig} \\ \hline
		20 & 1 & 001 \\
	\end{tabular}\\

Die komplette Bezeichnung der Anforderung würde wie folgt aussehen. \\

[201001] Datenbank: Design

\subsection{Systemanforderungen}

Die App wurde für ein Samsung Galaxy S5 mit Android 5.0 entwickelt. Aufgrund der verwendeten Android Layouts,
 sollte die App allerdings ohne Probleme auf allen Android Smartphones und Tablets mit mindestens Android 5.0 
 funktionieren.
\section{Übersicht}

\begin{table} [htbp]
	\begin{tabular}{|c|c|l|}
			\rowcolor{black} {\color{white}\textbf{ID}} & {\color{white}\textbf{Prio}} & {\color{white}\textbf{Anforderung}} \\ \hline
				101001 & 1 & Item manuell anlegen \\ \hline
				\rowcolor{DarkSeaGreen} 101002 & 1 & Item Information bearbeiten \\ \hline
				101003 & 1 & Item löschen \\ \hline
				\rowcolor{DarkSeaGreen} 102001 & 2 & Filter für die Sammlung \\ \hline
				102002 & 2 & Verleihverwaltung \\ \hline
				\rowcolor{DarkSeaGreen} 102003 & 2 & Alle Items exportieren (xml) \\ \hline
				102004 & 2 & Alle Item sexportieren (csv) \\ \hline
				\rowcolor{DarkSeaGreen} 102005 & 2 & Datenbank Administration implementieren \\ \hline
				103001 & 3 & REST Schnittstelle für Metadaten Abfrage diverser Online Datenbanken \\ \hline
				\rowcolor{DarkSeaGreen} 104001 & 4 & Kamera als Barcodescanner \\ \hline
				201001 & 1 & Datenbank Design \\ \hline
				\rowcolor{DarkSeaGreen} 201002 & 1 & Datenbank in App erstellen \\ \hline
				301001 & 1 & GUI: Item bearbeiten \\ \hline
				\rowcolor{DarkSeaGreen} 302001 & 2 & GUI: Items exportieren \\ \hline
				302002 & 2 & GUI: Liste verliehener Items \\ \hline
				\rowcolor{DarkSeaGreen} 304001 & 4 & GUI: Kamera als Barcodescanner \\ \hline
	\end{tabular}
\end{table}
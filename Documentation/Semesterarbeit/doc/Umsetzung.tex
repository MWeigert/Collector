\chapter{Umsetzung}
\label{ch:Umsetzung}

Um die komplette Umsetzung im Detail zu zeigen, fehlt der Platz und die Zeit. Zusätzlich würde dies, das vorliegende Dokument extrem Unübersichtlich machen. Deshalb wird im folgenden die Funktionsweise und Umsetzung an einigen typischen Beispielen erklärt. Alle im Laufe der Semesterarbeit erstellten Dateien inklusive des kompletten Quellcode kann man auf \href{https://github.com/MWeigert/Collector}{Github}\footnote{https://github.com/MWeigert/Collector} finden.\\

Wer sich den Quellcode genauer ansieht, wird entdecken, das von Anfang an ein Boolean (debugMode) übergeben wird. Diesen Boolean habe ich genutzt um an vielen Stellen einen Eintrag ins Logfile zu generieren, welcher es mir ermöglicht hat ohne aus der laufenden App auszusteigen den Verlauf der Variablen und Objekte jederzeit nachzuvollziehen.

\section{Funktionen}

An dieser Stelle hätte ich gerne mehr gezeigt. Dieser Abschnitt sollte zwei für die Useability der App entscheidende Funktionen beinhalten. 

\begin{itemize}
	\item Die Kamera als Barcodescanner.
	\item Eine REST Schnittstelle zur Metadaten Abfrage im Internet.
\end{itemize}

Wie es sich bei unserem Design Review abgezeichnet hatte, konnte ich diese beiden in der Anforderung und im Konzept beschrieben Funktionalitäten leider nicht mehr realisieren.

\subsection{Filtern der Sammlung}

Technisch habe ich zwei bereits eingesetzte Klassen verwendet um eine einfache und leicht zu erweiternde Filterfunktion in der App zu realisieren. Ich habe die Klasse \texttt{DatabaseOperations} um eine Methode erweitert, welche mir auf ein übergebenes Objekt der Klasse Item ein SELECT Statement liefert zum finden der Parameter des Items.

\lstinputlisting[frame=single,language=java,label=selectStatement,caption=Methode: getSelectStatement]{code/selectStatement.java}

Dazu habe ich wie im Listing \ref{selectStatement} ersichtlich ist, die Klasse Item als Container für die vom User eingegeben Parameter genutzt und aus diesen wird ein SELECT Statement zusammengesetzt. Pro Parameter wird ein Substring mit einer AND Verknüpfung angehängt. Zum Schluss entferne ich noch das letzte AND und schliesse das SELECT Statement mit \glqq;\grqq.\\

Der String des zurück gegeben SELECT Statement kann nun einfach von einem Intent zum anderen weiter gegeben werden. Im Anzeige Layout der Sammlung können nun entweder alle oder nur die Items der Sammlung in der Liste dargestellt werden, welche dem SELECT Statement entsprechen.

\subsection{Export der Daten}

Um die Daten der SQLite Datenbank zu exportieren habe ich die frei Verfügbare Library Opencsv (Version 3.7) eingebunden. Der Export erfolgt auf zwei verschiedene Arten

\begin{itemize}
	\item die vollständige Tabelle Items
	\item die gefilterte Tabelle Items
\end{itemize}

und wird von der Methode \texttt{exportDatabaseCSV} aus Listing \ref{exportCSV} ausgeführt.

\lstinputlisting[frame=single,language=java,label=exportCSV,caption=Methode: exportDatabaseCSV]{code/exportCSV.java}

Dieser Methode kann neben dem Datenbank Object ein String mit einem SELECT Statement übergeben werden und je nachdem ob ein String übergeben wird oder nicht schreibt die Methode die gefilterte oder ungefilterte Sammlung an Items in den externen Gerätespeicher. Nach erfolgreichem Export startet die Klasse welche den Export aufgerufen hat eine Intent, welcher das lokale E-Mail Programm ausführt und diesem die gespeicherte CSV Datei als Attachment übergibt.\\

Aufgerufen wird die Methode entweder über das Hauptmenu (Export aller Items) oder über die Filteransicht (Export der Items, welche den Filterkriterien genügen).

\section{Datenbank}

\subsection{Aufbau der SQLITE Datenbank}

Die Datenbank der App hat folgende Tabellen und Felder.

\begin{table} [htbp]
	\begin{center}
		\begin{tabular}{|l|l|l|}
			\rowcolor{black} {\color{white}\textbf{}} & {\color{white}\textbf{Typ}} & {\color{white}\textbf{Key}} \\
			\textbf{Items} & &  \\ \hline
			\rowcolor{DarkSeaGreen} ITEM\_ID & Integer & Ja \\ \hline
			EAN & Integer & \\ \hline
			\rowcolor{DarkSeaGreen} TITLE & String & \\ \hline	
			RATING & Integer & \\ \hline
			\rowcolor{DarkSeaGreen} MEDIA\_TYPE & String & \\ \hline
			GENRE\_ID & Integer & \\ \hline
			\rowcolor{DarkSeaGreen} LANGUAGE\_ID & Integer & \\ \hline
			LAUNCH & String \\ \hline
			\rowcolor{DarkSeaGreen} RENTAL & Integer & \\ \hline
			PUBLISHER\_ID & Integer & \\ \hline
			\rowcolor{DarkSeaGreen} AUTHOR\_ID & Integer & \\ \hline
			SYSTEM\_ID & Integer & \\ \hline
			\rowcolor{DarkSeaGreen} DVD & Integer & \\ \hline
			BLURAY & Integer & \\ \hline
			\rowcolor{DarkSeaGreen} STUDIO\_ID & Integer & \\ \hline
			DIRECTOR\_ID & Integer & \\ \hline
			\rowcolor{DarkSeaGreen} PARENTAL & Integer & \\ \hline
			REMARKS & String & \\ \hline
			\rowcolor{DarkSeaGreen} \textbf{History} & & \\ \hline
			HISTORY\_ID & Integer & Ja \\ \hline
			\rowcolor{DarkSeaGreen} ITEM\_ID & Integer & \\ \hline
			FRIEND\_ID & Integer & \\ \hline
			\rowcolor{DarkSeaGreen} START & String & \\ \hline
			BACK & String & \\ \hline
			\rowcolor{DarkSeaGreen} \textbf{Friends} & & \\ \hline
			FRIEND\_ID & Integer & Ja \\ \hline
			\rowcolor{DarkSeaGreen} FIRST\_NAME & String & \\ \hline
			LAST\_NAME & String & \\ \hline
			\rowcolor{DarkSeaGreen} \textbf{Authors} & & \\ \hline
			AUTHOR\_ID & Integer & Ja \\ \hline
			\rowcolor{DarkSeaGreen} AUTHOR & String & \\ \hline 
		\end{tabular}
		\caption{Datenbankfelder}
		\label{tab:Datenbankfelder}
	\end{center}
\end{table}

\newpage

\begin{landscape}
	\subsection{Übersicht Datenbank}
	\label{subsec:UebersichtDB}
	\begin{figure}[htbp]
		\centering
		\includegraphics[scale=0.6]{pic/DbDesign}
		\caption{Überblick Datenbank}
	\end{figure}
\end{landscape}

\section{GUI}

Allgemein wird von der GUI erwartet, dass die App intuitiv bedienbar ist und fehlerfrei auf möglichst vielen Hardware Typen dargestellt wird. \\

Des weiteren ist die GUI so designt das die App jederzeit übersichtlich ein Maximum an Informationen darstellt.

\subsection{Startseite}

Die Startseite der App soll dem User ermöglichen auf alle wichtigen Funktionen direkt zuzugreifen. Deshalb besteht diese nur aus folgenden vier Buttons.

\begin{enumerate}
	\item Neues Item
	\item Sammlung
	\item Leihwesen
	\item Info
\end{enumerate}

\begin{figure}[htbp]
	\centering
	\includegraphics[scale=0.5]{pic/GUI/Main}
	\caption{Startseite}
\end{figure}

\section{Fazit}

Die Semesterarbeit hat mir viel Freude bereitet und ich kann sagen, dass die Erfahrungen, welche ich auf dem Weg der Fertigstellung gesammelt habe mich wachsen haben lassen. Ganz klar muss ich eingestehen, dass meine mangelnde Erfahrung in der Software Entwicklung, vor allem im Bereich Android Applikationen, es mir deutlich schwerer gemacht hat, als ich am Anfang dachte und geplant hatte.\\

Viele in der Planungsphase scheinbare einfache User Stories haben mich letztendlich in der Implementierung ein vielfaches der dafür vorgesehenen Zeit gekostet. Auf der anderen Seite, hat die Arbeit an der App geholfen einige mir bisher nur theoretisch bekannte Arbeitsweisen praktisch nahe zu bringen. Nun am Ende der Entwicklung, bemerke ich eine deutliche Zunahme, was mein Wissen betreffend der Software Entwicklung für Android Geräte angeht.\\

Würde ich mit meinem heutigen Wissen die gleiche Aufgabe nochmals angehen, dann würde ich einiges anders machen und wäre diesmal sicher, dass ich in der mir zur Verfügung stehenden Zeit alle Anforderungen fristgerecht umsetzen könnte.\\

Vor allem im Bereich der Objekt Orientierten Programmierung haben sich mir, leider erst recht spät im Projekt, einige grundlegende Mechanismen erschlossen, welche mir in der Theorie bekannt waren, ich aber bisher nicht verinnerlicht hatte. Dies kann man sicher am Code sehen, der am Anfang noch sehr Prozedural ist und erst gegen Ende Objekt Orientierte Ansätze zeigt.\\

Durch diese anfänglichen Schwierigkeiten, habe ich mir viel unnötige Arbeit gemacht und unzählige überflüssige Zeilen Code generiert.\\

Trotz all den Schwierigkeiten und den Abstrichen an der Umsetzung der Anforderungen, muss ich gestehen bin ich darauf stolz das ich am Ende, auch dank der einmaligen Fristverlängerung, eine stabile lauffähige App erstellt habe, die zwar nicht alle gewünschten Features beinhaltet aber so hoffe ich die minimal Anforderung dieser Arbeit erfüllt.
\section{GUI}
\label{sec:GUIumsetzung}

Die GUI, der App wird gemäss dem MVC Pattern (Abb. \ref{fig:MVC}) umgesetzt. Der Aufbau von Android unterstützt den Entwickler sich an dieses Pattern zu halten.

\begin{figure}[htbp]
	\centering
	\includegraphics[scale=0.6]{pic/MVC}
	\caption{Modell, View \& Controller Pattern}
	\label{fig:MVC}
\end{figure}

Die Struktur, welche einem von Android vorgegeben wird sieht bei der App wie folgt aus:

\begin{itemize}
	\item Controller - Java Files welche die Funktion beinhalten und die Kommunikation zwischen View und Modell steuern.
	\item View - XML Files welche ausschliesslich das Aussehen der App definieren.
	\item Modell - SQLite Datenbank, welche die Daten für die App beinhaltet.
\end{itemize}

Wie das Pattern umgesetzt wurde anhand einzelner Ausschnitte aus dem Code der App. Auf einen Ausschnitt aus der SQLite Datenbank (dem Modell), wird an dieser Stelle verzichtet. Die Verwendete Datenbank Struktur zeigt die Abbildung \ref{subsec:UebersichtDB}. Sehr schön sieht man das Zusammenspiel zwischen Controller und View an der Zeile 3 \& 4 im Ausschnitt aus dem Button Listener \ref{listener} und der Zeile 2 im Ausschnitt des XML Files \ref{button}.\\

Hier sieht man im Controller eine Switch Verzweigung, welche anhand der übergebenen Button Id (btn\_mng\_author) eine Aktion ausführt. Welche ein neues Layout startet in welchem die Daten aus dem Modell \ref{listItem} angezeigt. 

\lstinputlisting[frame=single,language=java,label=listener,caption=Controller: Button Listener]{code/listenerDB.java}

\lstinputlisting[frame=single,language=xml,label=button,caption=View: Button im XML Layout]{code/button.xml}

\lstinputlisting[frame=single,language=java,label=listItem,caption=Controller: Daten aus Modell in View]{code/listItem.java}
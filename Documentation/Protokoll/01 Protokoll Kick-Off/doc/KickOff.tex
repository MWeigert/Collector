\begin{tabular}{ll}
	\textbf{Protokoll:} & Kick-Off Meeting Seminararbeit \\
	\textbf{Anwesend:} & R. Knaack, M. Reiser, M. Weigert \\
	\textbf{Protokollführer:} & M. Weigert \\
\end{tabular}

\begin{enumerate}
	\item Steht die Auftraggeberin bzw. der Auftraggeber hinter dieser Semesterarbeit? \\
	{\color{DarkSlateBlue}Ja, steht hinter der Semesterarbeit.}
	\item Sind die fachliche Kompetenz und die Verfügbarkeit der Betreuungsperson sichergestellt? \\
	{\color{DarkSlateBlue}Ja, kein Problem.}
	\item Sind die Urheberrechte und Publikationsrechte geklärt? \\
	{\color{DarkSlateBlue}Liegen bei M. Weigert.}
	\item Bekommt die Studentin oder der Student die notwendige logistische und beratende Unterstützung
	durch die Auftraggeberin bzw. den Auftraggeber? \\
	{\color{DarkSlateBlue}Nicht relevant.}
	\item Entsprechen Thema und Aufgabenstellungen den Anforderungen an eine Semesterarbeit? \\
	{\color{DarkSlateBlue}Der Umfang ist Ok, aber nicht gross.}
	\item Ist die Arbeit thematisch klar abgegrenzt und terminlich entkoppelt von den Prozessen
	(des Unternehmens) der Auftraggeberin bzw. des Auftraggebers? \\
	{\color{DarkSlateBlue}Nicht relevant.}
	\item Ist eine Grobplanung vorhanden? Sind die nächsten Schritte klar formuliert (von der
	Studentin oder dem Studenten)? \\
	{\color{DarkSlateBlue}Die Grobplanung besteht bereits. Feinplanung wird in den nächsten Tagen fertiggestellt.}
	\item Ist die Arbeit technisch und terminlich von der Studentin oder dem Studenten umsetzbar? \\
	{\color{DarkSlateBlue}Ja.}
\end{enumerate}

\textbf{M. Reiser:}
\begin{itemize}
	\item Umfang der Semesterarbeit hat Minimum Grösse und sollte nicht reduziert werden.
	\item Eine gute Dokumentation ist wegen der Grösse absolutes muss.
	\item Alle Punkte müssen komplett funktionsfähig sein.
	\item Die Requirementsanalyse beachten und gut ausarbeiten.
\end{itemize}

\textbf{R. Knaack}
\begin{itemize}
	\item Jetzt 1-2 Stunden für Feinplanung investieren und Dokumentieren.
	\item Ein Excel für die Zeitplanung erstellen mit Soll \& Ist Spalte. Dieses Tabelle sauber 
	bis zum Ende der Semesterarbeit führen (auch im Hinblick auf Planung Bachelorarbeit).
	\item Alle zwei Wochen Statusupdate an Herr Reiser senden.
	\item Bei der Semesterarbeit liegt der Schwerpunkt auf der Umsetzung.
	\item Der Umfang der Semesterarbeit ist, betrachtet man die Vorkenntnisse von M. Weigert durchaus angemessen.
	\item Denk Fokus auf eine genügende und termingerechte Arbeit legen.
	\item Sobald das Protokoll und die Feinplanung fertig gestellt ist den Termin für Design Review buchen. 
\end{itemize}
\begin{tabular}{ll}
	\textbf{Protokoll:} & Design Review Seminararbeit \\
	\textbf{Datum:} & 16. März 2016 \\
	\textbf{Anwesend:} & R. Knaack, M. Reiser, M. Weigert \\
	\textbf{Protokollführer:} & M. Weigert \\
\end{tabular}

\begin{enumerate}
	\item Ist der Auftrag für die Semesterarbeit von der Studentin oder dem Studenten korrekt erfasst? \\
	{\color{DarkSlateBlue}Ja}
	\item Sind die Ausgangslage und das Umfeld ausreichend analysiert, berücksichtigt und bearbeitet? \\
	{\color{DarkSlateBlue}Anforderungsanalyse muss verfeinert werden und Feinplanung Zeitmanagement für restliche Arbeit definiert.}
	\item Wurde eine systematische Recherche durchgeführt?\\
	{\color{DarkSlateBlue}Ja}
	\item Wurde ein klares Konzept erarbeitet und klar dargestellt?\\
	{\color{DarkSlateBlue}Es wurde kein Konzept, keine UseCases oder Workflow gezeigt. Zusätzlich sollen noch die Verwendeten Design Patterns und die Gründe für die Entscheidung dokumentiert werden.}
	\item Wurden alternative Lösungen betrachtet? \\
	{\color{DarkSlateBlue}Es sollen nicht Verwendete Design Pattern, als alternative Lösungswege integriert werden.}
	\item Entsprechen das Arbeits- und Lösungskonzept den Anforderungen an eine Semesterarbeit? \\
	{\color{DarkSlateBlue}Ja}
	\item Ist die Lösung grundsätzlich für die Auftraggeberin bzw. den Auftraggeber akzeptabel? \\
	{\color{DarkSlateBlue}Nicht relevant.}
	\item Ist das Konzept bzw. die Lösung technisch und terminlich im Rahmen der verbleibenden Zeit umsetzbar? \\
	{\color{DarkSlateBlue}Es wird von M. Weigert eine Feinplanung für das restliche Projekt inklusive Zeitbudget erstellt. Anhand diesem kann dann die Frage beantwortet werden.}
	\item Sind die nächsten Arbeitsschritte klar formuliert?\\
	{\color{DarkSlateBlue}Ja
		\begin{enumerate}
			\item Überarbeiten und priorisieren der Anforderungen.
			\item Feinplanung der UserStories und Meilensteine inklusive Zeitbudget und Terminplanung (bis 23.3.2016)
			\item Prototyp der App fertigstellen (bis 31. März 2016)
			\item App mit allen Anforderungen fertigstellen (bis 17. April 2016)
			\item Testing, Bug Fixing und fertigstellen der Dokumentation (bis 1. Mai 2016)
			\item Abgabe der Semesterarbeit am 3. Mai 2016
		\end{enumerate}}
\end{enumerate}

\newpage

Stichwortartig der Input von Herrn M. Reiser und R. Knack zum aktuellen Stand der Semesterarbeit.\\

\textbf{M. Reiser:}
\begin{itemize}
	\item UseCases in die Anforderungsanalyse übernehmen.
\end{itemize}

\textbf{R. Knaack}
\begin{itemize}
	\item Anforderungsanalyse anhand der Standards (s.h. Vorlesung M. Baumann) überarbeiten und dokumentieren.
	\item Anforderungen priorisieren und Umsetzung planen.
	\item Im Dokument mittels Querverweise dokumentieren, wie die Anforderungen umgesetzt wurden.
	\item M. Weigert soll sich im 1 bis 2 Wochenrhythmus unaufgefordert bei Herrn M. Reiser melden und die Fortschritte und Abweichungen von der Planung aufzeigen.
	\item Projektplanung eventuell über Gantt Chart realisieren, wobei der Umfang einer Semesterarbeit fast zu klein ist.
	\item Der Fokus der Semesterarbeit liegt klar auf der praktischen Anwendung der App.
	\item Für zukünftige Design Reviews den Fokus bei der Präsentation mehr auf die Funktion legen. Diesen in der Präsentation auch vorführen.  
\end{itemize}
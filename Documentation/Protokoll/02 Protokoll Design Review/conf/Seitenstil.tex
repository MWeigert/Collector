%----------------------------------------------------------------------------------------

% Formatvorlage für Arbeiten von Mathias Weigert (Grundlage der Datei und allg. Struktur
% von Micha Schöneberger)

%--------------------------------------VERSION-------------------------------------------

% Version V0.30 - Erstellt von Mathias Weigert am 27.08.2013
%				  				In erster Linie etwas aufgeräumt und übersichtlicher
% Version V0.75 - Erstellt von Mathias Weigert am 16.09.2013
%				  				Etwas aufgeräumt, muss mal bei Zeit detailiert überarbeitet werden
% Version V0.80 - Erstellt von Mathias Weigert am 24.08.2015
%									In der Kopfzeile wird nun nur noch das Kapitel angezeigt. Fusszeile
%									zeigt Version, Pfad und Seitenzahl an.

%--------------------------------ZEILENABSTAND--------------------------------------------

%\singlespacing		% Normaler Zeilenabstand
\onehalfspacing		% 1.5 facher Zeilenabstand [Standard]
%\doublespacing		% Doppelter Zeilenabstand

%-----------------------------------------------------------------------------------------

%----------------------------------SEITENRÄNDER-------------------------------------------

\geometry{
paper=a4paper,		% Papierformat			[a4paper]
left=20mm,			% Linker Seitenrand		[35mm]
right=20mm,			% Rechter Seitenrand	[25mm]
top=20mm,	 		% Oberer Seitenrand		[10mm]
bottom=48mm			% Unterer Seitenrand	[48mm]
}

%-----------------------------------------------------------------------------------------

% ----------------------------KOPF- UND FUSSZEILE-----------------------------------------

\pagestyle{scrheadings}
% Kopf- und Fußzeile auch auf Kapitelanfangsseiten
\renewcommand*{\chapterpagestyle}{scrheadings}
% definiert die Schriftart für die Kopfzeile (z.B: auch möglich: \sffamily)
\renewcommand{\headfont}{\normalfont}

% ================================KOPFZEILEN DESIGN======================================-

% \ohead[Auf Kapitelstartseiten oben außen]{Auf allen anderen Seiten oben außen}
% \chead[Auf Kapitelstartseiten oben mitte]{Auf allen anderen Seiten oben mitte}
% \ihead[Auf Kapitelstartseiten oben innen]{Auf allen anderen Seiten oben innen}

% --------------------------------KOPFZEILEN INNEN----------------------------------------

%\ihead{\huge{{\color{MediumBlue}\file}}}
\ihead{\huge{{\color{MediumBlue}\headmark}}}

% --------------------------------KOPFZEILEN MITTE----------------------------------------

\chead{}

% --------------------------------KOPFZEILEN AUSSEN---------------------------------------

\ohead{}												% Rechte Kopfzeile

\setlength{\headheight}{25mm} 	% Höhe der Kopfzeile
\setheadsepline[text]{1pt} 			% Trennlinie unter Kopfzeile

% ========================================================================================

% --------------------------------FUSSZEILEN DESIGN---------------------------------------

% \ofoot[Auf Kapitelstartseiten oben außen]{Auf allen anderen Seiten oben außen}
% \cfoot[Auf Kapitelstartseiten oben mitte]{Auf allen anderen Seiten oben mitte}
% \ifoot[Auf Kapitelstartseiten oben innen]{Auf allen anderen Seiten oben innen}

% --------------------------------FUSSZEILEN INNEN----------------------------------------

\ifoot{\\
	Protokoll: \version \\
	\tiny{\path \file}
}

% --------------------------------FUSSZEILEN MITTE----------------------------------------

\cfoot{}

% --------------------------------FUSSZEILEN AUSSEN---------------------------------------

\ofoot{Seite \pagemark \ von \pageref{LastPage}}

\setfootsepline[text]{1pt} 		% Trennlinie über Fusszeile

%-----------------------------------------------------------------------------------------

% --------------------------SCHUSTERJUNGEN UND HURENKINDER--------------------------------

\clubpenalty = 10000			% Disable single lines at the start of a paragraph
\widowpenalty = 10000 			% Disable single lines at the end of a paragraph
\displaywidowpenalty = 10000	% Disable single lines at the end of a paragraph

% ------------------------------------ALLGEMEINES-----------------------------------------

\frenchspacing							% erzeugt ein wenig mehr Platz hinter einem Punkt
\counterwithout{footnote}{chapter}		% Fußnoten fortlaufend durchnummerieren

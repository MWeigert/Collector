\section{Ausgangssituation}
\frame{\frametitle{Ausgangssituation}
Immer wieder komme ich in die Situation das ich auf Flohmärkten und in Brockenhäusern Bücher, Filme oder
Konsolenspiele kaufe nur um später zu Hause festzustellen, dass ich diese bereits besitze.\\[2ex]

Deshalb habe ich nach einer Lösung gesucht, Medien aus diesen Kategorien einfach zu verwalten.\\[2ex]

Im Google Play Store konnte ich die ein oder andere App finden, welche Sammlungen aus einer der Kategorien
verwalten kann, aber keine welche in der Lage ist Bücher, Filme und Konsolenspiele zu verwalten.
}

\frame{\frametitle{Ausgangssituation}
	\begin{table}	
		\begin{tabular}{|l|c|c|c|c|}
			\hline
			\textbf{Name} & \textbf{Bücher} & \textbf{Filme} & \textbf{Spiele} & \textbf{Scanner} \\ \hline
			My books & Ja & \color{red}Nein & \color{red}Nein & Ja \\ \hline
			Meine Bücher Pro & Ja & \color{red}Nein & \color{red}Nein & Ja \\ \hline
			Codex & Ja & \color{red}Nein & \color{red}Nein & Ja \\ \hline
			Movie Collection & \color{red}Nein & Ja & \color{red}Nein & Ja \\ \hline
			Movielicious & \color{red}Nein & Ja & \color{red}Nein & Ja \\ \hline
			Nintendo Collection & \color{red}Nein & \color{red}Nein & Ja (NES) & \color{red}Nein \\ \hline
		\end{tabular} 
	\end{table}
}

\frame{
	\begin{table}	
		\begin{tabular}{|l|c|c|c|c|c|c|r|c|}
			\hline
			\textbf{Name} & \textbf{Export} & \textbf{Verleih} & \textbf{Preis} & \textbf{Rating} \\ \hline
			My books & Ja & Ja & Gratis & 3.7 \\ \hline
			Meine Bücher Pro & CSV & \color{black}Nein & CHF 3.65 & 4.0 \\ \hline
			Codex & CSV, XML & Ja & Gratis & 4.2 \\ \hline
			Movie Collection & CSV, XML & Ja & CHF 2.15 & 4.5 \\ \hline
			Movielicious & Ja & Ja & CHF 2.50 & 3.2 \\ \hline
			Nintendo Collection & \color{red}Nein & \color{red}Nein & CHF 1.47 & 5.0 \\ \hline
		\end{tabular} 
	\end{table}
}
%----------------------------------------------------------------------------------------

% Formatvorlage für Arbeiten von Mathias Weigert (Grundlage der Datei und allg. Struktur
% von Micha Schöneberger)

%--------------------------------------VERSION-------------------------------------------

% Version V0.10 - Erstellt von Mathias Weigert am 08.01.2014
%									Basis war die LaTex Dokumente der ZHAW
% Version V0.80 - Erstellt von Mathias Weigert am 24.08.2015
%									Versionsprung, damit Hauptdokument und Configurationsfiles nun selbe
%									Version besitzen. Alle benötigten Files sind nun in Unterverzeichnisse
%									ausgelagert.

% ---------------------------------LADEN DER META DATEN-----------------------------------

% Informationen über das Dokument, wie z.B. Titel, Autor, Jahr etc. werden in der
% Datei _Meta.tex definiert und können danach global verwendet werden.

%----------------------------------------------------------------------------------------

% Formatvorlage für Arbeiten von Mathias Weigert (Grundlage der Datei und allg. Struktur
% von Micha Schöneberger)

%--------------------------------------VERSION-------------------------------------------

% Version V0.75 - Erstellt von Mathias Weigert am 16.09.2013
%				  Bisher ungenützten "Balast" entfernt und zweite Matrikelnummer
%				  eingefügt.

%----------------------------------- INFORMATIONEN ---------------------------------------

% 	Definition von globalen Parametern, die im gesamten Dokument verwendet
% 	werden können (z.B auf dem Deckblatt etc.).
%-----------------------------------------------------------------------------------------

\newcommand{\titel}{Collector}	            % Titel des Dokumentes
\newcommand{\untertitel}{Semesterarbeit}	% Untertitel des Dokumentes
\newcommand{\autorFirst}{Mathias Weigert}	% Autor
\newcommand{\jahr}{2015}					% Jahr der Arbeit

\newcommand{\version}{Kick Off Semesterarbeit}				% Dokument Version
\newcommand{\gueltig}{2015}				    % Dokument Gültigkeit
\newcommand{\initial}{mwe}					% Initial des Autors
\newcommand{\freigabe}{}					% Freigegeben durch

\newcommand{\path}{D:\textbackslash Data\textbackslash Studium\textbackslash WorkSpace\textbackslash Collector\textbackslash Protokoll\textbackslash}	% Pfad des Files
\newcommand{\file}{KickOff.pdf}    	       % Name des Files

%-----------------------------------------------------------------------------------------

% ----------------------------------------------------------------------------------------

% ----------------------------------DOKUMENTENKOPF----------------------------------------

% Diese Vorlage basiert auf "scrreprt" aus dem koma-script.
% Die Option "draft" sollte beim fertigen Dokument ausgeschaltet werden.

\documentclass[
	9pt,				% Schriftgröße
	ngerman,			% für Umlaute, Silbentrennung etc.
	a4paper,       		% Papierformat
	oneside,			% einseitiges Dokument
	titlepage,			% es wird eine Titelseite verwendet
	final				% Status des Dokuments (final/draft)
]{scrbook}				% KOMA Script-Klassen: scrbook, scrreprt, scrartcl, scrlttr2.
						% Die Angabe einer Dokumentenklasse ist obligatorisch.

% ----------------------------------------------------------------------------------------

% -----------------------------PACKAGES EINLESEN------------------------------------------
% Weitere Packages, die benötigt werden, sind in die Datei Packages.tex "ausgelagert", um
% die Vorlage möglichst Übersichtlich zu halten.

%----------------------------------------------------------------------------------------

% Formatvorlage für Arbeiten von Mathias Weigert (Grundlage der Datei und allg. Struktur
% von Micha Schöneberger)

%--------------------------------------VERSION-------------------------------------------

% Version V0.75 - Erstellt von Mathias Weigert am 16.09.2013
%				  Etwas aufgeräumt, muss mal bei Zeit detailiert überarbeitet werden

%---------------------------------- INFORMATIONEN ---------------------------------------

% 	Definition von globalen Parametern, die im gesamten Dokument verwendet
% 	werden können (z.B auf dem Deckblatt etc.).
%----------------------------------------------------------------------------------------

\usepackage{cite}					% Verbesserte Zitierungsmöglichkeiten
\usepackage{tocbasic}

%----------------------- FORMATIERUNG KOPF- UND FUSSZEILE -------------------------------

\usepackage
[automark,							% Automatische Kopfzeile
%headtopline,						% Linie über dem Seitenkopf
%plainheadtopline,					% Plain, Linie über dem Seitenkopf
headsepline,						% Linie zwischen Kopf und Textkörper
ilines,								% Trennlinie linksbündig ausrichten
%plainheadsepline,					% Plain, Linie zwischen Kopf und Textkörper
footsepline,						% Linie zwischen Textkörper und Fuss
plainfootsepline,   				% Plain, Linie zwischen Textkörper und Fuss
%footbotline,						% Linie unter dem Fuss
%plainfootbotline   				% Plain, Linie unter dem Fuss
]{scrpage2}

%----------------------------------------------------------------------------------------


%-------------------------- ANPASSUNG LANDESSPRACHE--------------------------------------

\usepackage{ngerman}

%----------------------------------------------------------------------------------------


%--------------------------------- UMLAUTE-----------------------------------------------

\usepackage[T1]{fontenc}			% Dadurch ist auch sichergestellt, dass in einem PDF
									% Umlaute gefunden werden.
\usepackage[ansinew]{inputenc}		% Unterstützung erweiterter Zeichensätze mit
									%  unterschiedlichen Kodierungen
									% (z. B. ä, ö, ü usw.)
\usepackage{textcomp} 				% Text Companion Schriften (grosse Symbolbibliothek)
									
%----------------------------------------------------------------------------------------

\usepackage{pifont}					% PostScript standard Symbol and Dingbats fonts

%--------------------------------EIGENE REFERENZIERUNG-----------------------------------

\newcommand{\refTC}[1]{ \ref{#1} \nameref{#1}  }		
% gibt aus z.B.: 1.1.3 Erwartetes Resultat (<Kapitelnummer> <Kapitelname>)

%----------------------------------------------------------------------------------------


%----------------------------------- GRAFIKEN--------------------------------------------

\usepackage[dvips,final]{graphicx}		% Einbinden von EPS-Grafiken [draft oder final]
\graphicspath{{pic/}} 					% Hier liegen die Bilder des Dokuments

%----------------------------------------------------------------------------------------

\usepackage{amsmath,amsfonts}			% Befehle aus AMSTeX für mathematische Symbole
										% z.B. \boldsymbol \mathbb
\usepackage{multirow}					% Packages für Tabellen
	
%-------------------------------- OWN COLORS---------------------------------------------

\usepackage{conf/rgbcolor}				% Eigene Farbdefinitionen nach RGB Chart 
\usepackage[table]{xcolor}				% Um Tabellen einzuf�rben

%----------------------------------------------------------------------------------------

\usepackage{makeidx}					% Für Index-Ausgabe; \printindex

%------------------------------ QUELLCODE AUSGABE----------------------------------------

\input{conf/code}						% Eigene Ausgabeformatierungen für Quellcode

%----------------------------------------------------------------------------------------


%-------------------------- ZEILENABSTÄNDE UND SEITENRÄNDER------------------------------

\usepackage{setspace}					% hat zur Benützung 3 Optionen: [singlespacing,
										% onehalfspacing, doublespacing]
\usepackage{geometry}	
\usepackage{lineno}						% ermöglicht das Numerieren der Zeilen


% Symbolverzeichnis --------------------------------------------------------
% Symbolverzeichnisse bequem erstellen, beruht auf MakeIndex.
% makeindex.exe %Name%.nlo -s nomencl.ist -o %Name%.nls
% erzeugt dann das Verzeichnis. Dieser Befehl kann z.B. im TeXnicCenter
% als Postprozessor eingetragen werden, damit er nicht ständig manuell
% ausgeführt werden muss.
% Die Definitionen sind ausgegliedert in die Datei Abkuerzungen.tex.
% --------------------------------------------------------------------------
\usepackage[intoc]{nomencl}
  \let\abbrev\nomenclature
  \renewcommand{\nomname}{Abkürzungsverzeichnis}
  \setlength{\nomlabelwidth}{.25\hsize}
  \renewcommand{\nomlabel}[1]{#1 \dotfill}
  \setlength{\nomitemsep}{-\parsep}
  

%inserted november 2012 / Micha
\usepackage[normalem]{ulem}
\newcommand{\markup}[1]{\uline{#1}}


\usepackage{floatflt}							% Zum Umfließen von Bildern


% Zum Einbinden von Programmcode --------------------------------------------
%\usepackage{listings}
%\usepackage{xcolor} 
%\definecolor{hellgelb}{rgb}{1,1,0.9}
%\definecolor{colKeys}{rgb}{0,0,1}
%\definecolor{colIdentifier}{rgb}{0,0,0}
%\definecolor{colComments}{rgb}{1,0,0}
%\definecolor{colString}{rgb}{0,0.5,0}
%\lstset{%
%    float=hbp,%
%    basicstyle=\texttt\small, %
%    identifierstyle=\color{colIdentifier}, %
%    keywordstyle=\color{colKeys}, %
%    stringstyle=\color{colString}, %
%    commentstyle=\color{colComments}, %
%    columns=flexible, %
%    tabsize=2, %
%    frame=single, %
%    extendedchars=true, %
%    showspaces=false, %
%    showstringspaces=false, %
%    numbers=left, %
%    numberstyle=\tiny, %
%    breaklines=true, %
%    backgroundcolor=\color{hellgelb}, %
%    breakautoindent=true, %
%%    captionpos=b%
%}

\usepackage[hyphens]{url}			% bricht lange URL's um, ohne dass der Link defekt ist
\usepackage{url}

%-------------------------------PDF OPTIONEN----------------------------------------------
								
\usepackage[
bookmarks,
bookmarksopen=true,
pdftitle={\titel},
pdfauthor={\autorFirst},
pdfcreator={\autorFirst},
pdfsubject={\titel},
pdfkeywords={\titel},
colorlinks=true,
linkcolor=red, 							% einfache interne Verknüpfungen
anchorcolor=black,						% Ankertext
citecolor=blue, 			% Verweise auf Literaturverzeichniseinträge im Text
filecolor=magenta, 						% Verknüpfungen, die lokale Dateien öffnen
menucolor=red, 							% Acrobat-Menüpunkte
urlcolor=cyan,
%linkcolor=black, 						% einffdrtext
%citecolor=black, 			% Verweise auf Literaturverzeichniseinträge im Text
%filecolor=black, 						% Verknüpfungen, die lokale Dateien öffnen
%menucolor=black,						% Acrobat-Menüpunkte
%urlcolor=black,
backref,
%pagebackref,
plainpages=false,						% zur korrekten Erstellung der Bookmarks
pdfpagelabels,							% zur korrekten Erstellung der Bookmarks
hypertexnames=false,					% zur korrekten Erstellung der Bookmarks
linktocpage 				% Seitenzahlen anstatt Text im Inhaltsverzeichnis verlinken
]{hyperref}

\usepackage[final]{pdfpages}			%Ermöglicht andere PDF ins Dokument einzufügen.

%-----------------------------------------------------------------------------------------

\usepackage{chngcntr}					% Zum fortlaufenden Durchnummerieren der Fußnoten


% Aliase für Zitate
% \defcitealias{WPProzess}{Wikipedia:~Prozess}

%\usepackage{minitoc}

% für lange Tabellen
\usepackage{longtable}
\usepackage{array}
\usepackage{ragged2e}
\usepackage{lscape}

%Spaltendefinition rechtsbündig mit definierter Breite
\newcolumntype{w}[1]{>{\raggedleft\hspace{0pt}}p{#1}}

% Formatierung von Listen ändern
\usepackage{paralist}
% \setdefaultleftmargin{2.5em}{2.2em}{1.87em}{1.7em}{1em}{1em}	% lade ausgelagerte Packages

% ----------------------------------------------------------------------------------------

% -----------------------KOPF- UND FUSSZEILE, SEITENRÄNDER...-----------------------------
% Diese Einstellungen sind im Dokument Seitenstil ausgelagert.

%----------------------------------------------------------------------------------------

% Formatvorlage für Arbeiten von Mathias Weigert (Grundlage der Datei und allg. Struktur
% von Micha Schöneberger)

%--------------------------------------VERSION-------------------------------------------

% Version V0.3  - Erstellt von Mathias Weigert am 27.08.2013
%				  In erster Linie etwas aufgeräumt und übersichtlicher
% Version V0.75 - Erstellt von Mathias Weigert am 16.09.2013
%				  Etwas aufgeräumt, muss mal bei Zeit detailiert überarbeitet werden

%--------------------------------ZEILENABSTAND--------------------------------------------

%\singlespacing		% Normaler Zeilenabstand
\onehalfspacing		% 1.5 facher Zeilenabstand [Standard]
%\doublespacing		% Doppelter Zeilenabstand

%-----------------------------------------------------------------------------------------
			
%----------------------------------SEITENRÄNDER-------------------------------------------

\geometry{
paper=a4paper,		% Papierformat			[a4paper]
left=20mm,			% Linker Seitenrand		[35mm]
right=20mm,			% Rechter Seitenrand	[25mm]
top=20mm,	 		% Oberer Seitenrand		[10mm]
bottom=48mm			% Unterer Seitenrand	[48mm]
}

%-----------------------------------------------------------------------------------------

% ----------------------------KOPF- UND FUSSZEILE-----------------------------------------

\pagestyle{scrheadings}
% Kopf- und Fußzeile auch auf Kapitelanfangsseiten
\renewcommand*{\chapterpagestyle}{scrheadings}
% definiert die Schriftart für die Kopfzeile (z.B: auch möglich: \sffamily)
\renewcommand{\headfont}{\normalfont}

% --------------------------------KOPFZEILEN DESIGN---------------------------------------

% \ohead[Auf Kapitelstartseiten oben außen]{Auf allen anderen Seiten oben außen}
% \chead[Auf Kapitelstartseiten oben mitte]{Auf allen anderen Seiten oben mitte}
% \ihead[Auf Kapitelstartseiten oben innen]{Auf allen anderen Seiten oben innen}

\ihead{\textit{\headmark}}		% Kapitelname in der Kopfzeile darstellen
\chead{}
\ohead{\includegraphics[scale=0.08]{ZHAW_logo.png}} % Bild in Kopfzeile
\setlength{\headheight}{30mm} 	% Höhe der Kopfzeile
\setheadsepline[text]{0.4pt} 	% Trennlinie unter Kopfzeile

% --------------------------------FUSSZEILEN DESIGN---------------------------------------

% \ofoot[Auf Kapitelstartseiten oben außen]{Auf allen anderen Seiten oben außen}
% \cfoot[Auf Kapitelstartseiten oben mitte]{Auf allen anderen Seiten oben mitte}
% \ifoot[Auf Kapitelstartseiten oben innen]{Auf allen anderen Seiten oben innen}

\ifoot{\autorFirst}				% Autor in Fusszeile
\cfoot{}
\ofoot{\pagemark}				% Seitenzahl in Fusszeile
\setfootsepline[text]{0.4pt} 	% Trennlinie über Fusszeile

% --------------------------SCHUSTERJUNGEN UND HURENKINDER--------------------------------

\clubpenalty = 10000		% Disable single lines at the start of a paragraph
\widowpenalty = 10000 		% Disable single lines at the end of a paragraph
\displaywidowpenalty = 10000	% Disable single lines at the end of a paragraph

% ------------------------------------ALLGEMEINES-----------------------------------------

\frenchspacing							% erzeugt ein wenig mehr Platz hinter einem Punkt
\counterwithout{footnote}{chapter}		% Fußnoten fortlaufend durchnummerieren	% lade spezifische Kopf- und Fusszeilen

% ----------------------------------------------------------------------------------------

\begin{document}

\setlength{\parindent}{0pt}		% sorgt dafür das es keinen Einzug am Absatzanfang gibt
\setcounter{secnumdepth}{3}		% legt die Tiefe bei Nummerierung von Überschriften fest
\setcounter{tocdepth}{3}		% legt Schachtelungstiefe bei Inhaltsverzeichnis fest

% ----------------------------------------------------------------------------------------

% ---------------------DECKBLATT, INHALT & VERSIONSVERWALTUNG-----------------------------

%%----------------------------------------------------------------------------------------

% Formatvorlage für Arbeiten von Mathias Weigert (Grundlage der Datei und allg. Struktur
% von Micha Schöneberger)

%--------------------------------------VERSION-------------------------------------------

% Version V0.75 - Erstellt von Mathias Weigert am 16.09.2013
%				  Der besseren Übersicht wegen Informationen aus zwei Seiten verteilt

%------------------------------------SEITENLAYOUT-----------------------------------------

% Informationen Über das Dokument, wie z.B. Titel, Autor, Matrikelnr. etc werden in der
% Datei _Meta.tex definiert und können danach global verwendet werden.

%\thispagestyle{empty} 		% Dokument ohne Kopfzeile und ohne Seitennummerierung.
\thispagestyle{plain} 		% Mit Seitennummerierung (Standard)
%\thispagestyle{headings} 	% Generiert eine automatische Kopfzeile mit Seitenzahl und
							% Zwischenüberschriften
%\thispagestyle{myheadings} % Erlaubt die Erstellung eigener Kopf- und Fußzeilen
%\thispagestyle{fancy} 		% Erlaubt die Verwendung der in dem Paket "fancyhdr"
							% definierten Befehle zur Erstellung eigener Kopf- und
							% Fußzeilen

%-----------------------------------------------------------------------------------------

% ---------------------------------SETZE TITELBLATT---------------------------------------

\begin{titlepage}

\begin{center} 									% setze die Formatierung auf Center
\includegraphics[scale=0.25]{pic/ZHAW_Titel} 	% binde die Grafik ein
\\[12ex]
\huge{\textbf{\textsc{\titel}}}					% setze Schriftgrösse für Titel
\\[3ex]
\LARGE{\textbf{\untertitel}}					% setze Schriftgrösse für Untertitel
\\[6ex]					
\LARGE{\textbf{\art}}							% setze Schriftgrösse für Art der Arbeit
\\[4ex]
\Large{im Fachgebiet \fachgebiet}				% setze Schriftgrösse für Fachgebiet
\\[4ex]
\normalsize								% setze Schriftgrösse zurück auf Normalschrift	

\newpage										% zweite Seite mit allg. Informationen

\begin{tabular}{w{3cm}p{6cm}}
%\\[12ex]Studienbereich: & \quad \studienbereich	% Angabe des Studienbereichs
\\[12ex]
Student: & \quad \autorFirst					% Angabe des Studenten
\\[1.2ex]
% & \quad \autorSecond							% Angabe eines zweiten Studenten
\\[4ex]
Matrikelnummer: & \quad \matrikelnrFirst 		% Angabe der Matrikelnummer
\\[1.2ex]
%Matrikelnummer: & \quad \matrikelnrSecond 		% Angabe der Matrikelnummer
\\[4ex]
Dozent: & \quad \erstgutachter					% Angabe des Dozenten
\\[1.2ex]
Zweitgutachter: & \quad \zweitgutachter			% Angabe des Zweitgutachters
\\[3ex]
\end{tabular}
\\[48ex]
\copyright\ \jahr\\
\end{center}

\singlespacing						% einfacher Zeilenabstand
\small								% setze kleine Schrift
\noindent							% Sorgt am Anfang eines Absatzes dafür, daß die erste
									% Zeile nicht eingerückt wird.
Dieses Werk einschließlich seiner Teile ist \textbf{urheberrechtlich geschützt}. Jede Verwertung außerhalb der engen Grenzen des Urheberrechtsgesetzes ist ohne Zustimmung des Autors unzulässig und strafbar. Das gilt insbesondere für Vervielfältigungen, Übersetzungen, Mikroverfilmungen sowie die Einspeicherung und Verarbeitung in elektronischen Systemen.

\end{titlepage}		% Deckblatt

%\tableofcontents				% Inhaltsverzeichnis

%\chapter*{Version}

Angabe der aktuellen Version des Dokuments. Alle Änderungen, werden mit Versionsnummer, Datum, Autor und Informationen betreffend der Änderung dokumentiert.\\

\begin{tabular}{|c|l|c|l|}
	\rowcolor{black} {\color{white}\textbf{Version}} & {\color{white}\textbf{Datum}} & {\color{white}\textbf{Autor}} & {\color{white}\textbf{Bemerkung}} \\
	0.01 & 30.11.15 & M. Weigert & Initiales Dokument \\ \hline
	\rowcolor{DarkSeaGreen} 0.02 & 30.11.15 & M. Weigert & Aufgabe hinzugefügt \\ \hline
	0.03 & 02.12.15 & M. Weigert & Neue LaTex Konfiguration, Literaturverzeichnis, erste Recherche \\ \hline
	\rowcolor{DarkSeaGreen} 0.04 & 04.12.15 & M. Weigert & Kapitelstruktur angelegt. Recherche fortgeführt \\ \hline
	0.04 & 03.02.16 & M. Weigert & Zeitmanagement und Feinstruktur \\ \hline
	\rowcolor{DarkSeaGreen} 0.05 & 06.02.16 & M. Weigert & Neue Datenstruktur und Anforderungen dokumentiert. \\ \hline
\end{tabular}

\chapter*{Zeitmanagement}

\begin{tabular}{|c|l|c|c|l|}
	\rowcolor{black} {\color{white}\textbf{Datum}} & {\color{white}\textbf{Start}} & {\color{white}\textbf{Ende}} & {\color{white}\textbf{Zeit (min.)}} & {\color{white}\textbf{Bemerkung}} \\
	03.02.16 & 20:30 & 21:00 & 30 & Zeitmanagement und Feinplanung (Basis) \\ \hline
	\rowcolor{DarkSeaGreen} 06.02.16 & 11:00 & 12:45 & 105 & Angefangen mit der Anforderungsanalyse \\ \hline
	06.02.16 & 13:30 & 14:30 & 60 & Weiter an Anforderungsanalyse gearbeitet. \\ \hline
\end{tabular}	% Versionsverwaltung

% ----------------------------------------------------------------------------------------

% -----------------------------INHALT DES DOKUMENTS---------------------------------------

% Hier können jetzt die einzelnen Kapitel implementiert werden. Sie müssen in den
% entsprechenden .TEX-Dateien vorliegen. Die Dateinamen können natürlich angepasst werden.

% ----------------------------------------------------------------------------------------

\begin{tabular}{ll}
	\textbf{Protokoll:} & Design Review Seminararbeit \\
	\textbf{Anwesend:} & R. Knaack, M. Reiser, M. Weigert \\
	\textbf{Protokollführer:} & M. Weigert \\
\end{tabular}

\begin{enumerate}
	\item Ist der Auftrag für die Semesterarbeit von der Studentin oder dem Studenten korrekt erfasst? \\
	{\color{DarkSlateBlue}LOREM IPSUM DOLOR}
	\item Sind die Ausgangslage und das Umfeld ausreichend analysiert, berücksichtigt und bearbeitet? \\
	{\color{DarkSlateBlue}LOREM IPSUM DOLOR}
	\item Wurde eine systematische Recherche durchgeführt?\\
	{\color{DarkSlateBlue}LOREM IPSUM DOLOR}
	\item Wurde ein klares Konzept erarbeiotet und klar dargestellt?\\
	{\color{DarkSlateBlue}LOREM IPSUM DOLOR}
	\item Wurden alternative Lösungen betrachtet? \\
	{\color{DarkSlateBlue}LOREM IPSUM DOLOR}
	\item Entsprechen das Arbeits- und Lösungskonzept den Anforderungen an eine Semesterarbeit? \\
	{\color{DarkSlateBlue}LOREM IPSUM DOLOR}
	\item Ist die Lösung grundsätzlich für die Auftraggeberin bzw. den Auftraggeber akzeptabel? \\
	{\color{DarkSlateBlue}LOREM IPSUM DOLOR}
	\item Ist das Konzept bzw. die Lösung technisch und terminlich im Rahmen der verbleibenden Zeit umsetzbar? \\
	{\color{DarkSlateBlue}LOREM IPSUM DOLOR}
	\item Sind die nächsten Arbeitsschritte klar formuliert?\\
	{\color{DarkSlateBlue}LOREM IPSUM DOLOR}
\end{enumerate}

\textbf{M. Reiser:}
\begin{itemize}
	\item INSERT ITEM HERE
\end{itemize}

\textbf{R. Knaack}
\begin{itemize}
	\item INSERT ITEM HERE
\end{itemize}

% --------------------------------------ANHANG--------------------------------------------

% Die Inhalte des Anhangs werden analog zu den Kapiteln implementiert; über die Datei
% Anhang.tex

%\begin{appendix}
%	\clearpage
%	\pagenumbering{Roman}
%	\renewcommand{\thesection}{\arabic{section}}
%	\chapter{Anhang}

\section{Konventionen}

Für die Beschreibung von Eingaben oder Beschriftungen in der App wird folgendes Format benutzt. \\
{\color{IndianRed}\texttt{Beispiel einer Eingabe durch den User oder Beschriftung in der App.}}

\section{Verwendete Tools \& Software}

\subsection{Dokumentation \& Präsentation}

Alle Dokumente und Präsentationen wurden in \LaTeX \ geschrieben. Dazu wurde \href{http://texstudio.sourceforge.net}{TeXstudio} in der Version 2.10.4 genutzt. 

\subsection{Code und Versionsverwaltung}

Für die Versionsverwaltung aller Dokumente und des Programmcode wurde auf GitHub ein \href{https://github.com/MWeigert/Collector}{Repository} eingerichtet.

\subsection{GUI}

Für die Funktion und das Design der GUI wurde das Webtool \href{https://www.fluidui.com}{Fluid} genutzt.

\section{Design Entscheidungen}

\begin{tabular}{|c|l|c|l|}
	\rowcolor{black} {\color{white}\textbf{ID}} & {\color{white}\textbf{Datum}} & {\color{white}\textbf{Autor}} & {\color{white}\textbf{Bemerkung}} \\
	001 & 12.03.16 & M. Weigert & Hauptsprache für App wird Englisch \\ \hline
	\rowcolor{DarkSeaGreen} 002 & 12.03.16 & M. Weigert & Es wird auf jeden Zurück/Back Button verzichtet. \\ \hline
\end{tabular}
%	\bibliography{bib/mybib.bib}
%	\bibliographystyle{unsrt}
%\end{appendix}

% ----------------------------------------------------------------------------------------

\end{document}
